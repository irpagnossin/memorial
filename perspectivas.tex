\chapter{Perspectivas}

Eu tive a oportunidade de experimentar os meios acad�mico e corporativo, e a felicidade de trilhar o ``caminho do meio'', a despeito de todas as d�vidas, que sempre estiveram presentes. Muito foi realizado, especialmente no CEPA, e em grande parte sem a recompensa financeira que eu poderia ter obtido no meio corporativo. Mas esta � uma escolha que fa�o lucidamente, pois as minhas maiores recompensas s�o pessoais, e acima de tudo minha inten��o � construir algo de que me orgulhe e que contribua para o futuro do meu filho e da nossa sociedade. Eu tenho conseguido seguir este caminho, bem ou mal, com a ajuda de todas as pessoas que encontrei nessa jornada. Mas ela n�o tem fim...

Cria��o e produ��o de objetos de aprendizagem, e pesquisa sobre eles. Esses s�o os tr�s ingredientes que preciso. Atualmente tenho os dois primeiros, e j� come�o a lutar para conseguir o terceiro. � justamente onde este cargo de docente pode auxiliar. 

Independentemente disso, meu trabalho no CEPA continua, e para o pr�ximo semestre (at� o final do ano) tenho algumas metas bem claras:

\begin{compactitem}
	\item Finalizar a implanta��o do Scrum no CEPA;
	\item Buscar \foreign{feedback} dos alunos, tutores e educadores quanto aos Objetos de Aprendizagem disponibilizados;
	\item Difundir os Objetos de Aprendizagem entre os professores-autores, educadores e tutores (muitos deles sequer sabem que eles existem, e outros n�o imaginam que existe uma equipe capaz de realizar suas ideias);
	\item Difundir o padr�o SCORM entre os professores autores, educadores e tutores;
	\item Submeter dois ou tr�s trabalhos, em parceria com colegas acad�micos do CEPA, para os pr�ximos congressos da ABED (Associa��o Brasileira de Ensino a Dist�ncia), baseados em nossas experi�ncias nos projetos Aulas Interativas, Univesp e Redefor.
\end{compactitem}

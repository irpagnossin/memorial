\chapter{Perspectivas}

Eu n�o me considero um acad�mico, assim como eu n�o me considero um completo ser empresarial. Eu me considero meio-termo. 

Eu tenho feito muito trabalho pro bono no CEPA, mas n�o fa�o isso sem motivo. Minha inten��o � construir algo, e agora est� bastante claro para mim o que: agir na pesquisa, cria��o e produ��o desses softwares de ensino, seja para ensino a dist�ncia, seja para dispositivos m�veis ou para lousas eletr�nicas (ensino presencial). Apenas dois desses itens n�o ser�o suficientes; deve haver os tr�s. � onde entra o meu projeto de pesquisa, pois como docente da USP, terei como dedicar parte do meu tempo � pesquisa (atualmente ele todo devotado � cria��o e � produ��o desse materiais). Dito de outra forma: ter entrado para o CEPA foi o primeiro passo; ser docente da USP � um poss�vel segundo (mas certamente n�o �nico).

Independentemente disso, meu trabalho no CEPA continua, e para o pr�ximo semestre (at� o final do ano) tenho algumas metas bem clara:

\begin{compactitem}
	\item Finalizar a implanta��o do Scrum na equipe
	\item Buscar feedback dos alunos, tutores e educadores quanto as atividades disponibilizadas
	\item Procurar ao menos um tutor/educador disposto a criar atividades interativas com a equipe
	\item Difundir o padr�o SCORM para os participantes do projeto
	\item Escrever, juntamente com colegas do CEPA, dois ou tr�s artigos para a ABED, baseados nas experi�ncias nos projetos Aulas Interativas, UNIVESP e REDEFOR.
\end{compactitem}
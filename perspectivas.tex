\chapter{Perspectivas}

Eu tive a oportunidade de experimentar os meios acadêmico e corporativo, e a felicidade de trilhar o ``caminho do meio'', a despeito de todas as dúvidas, que sempre estiveram presentes. Muito foi realizado, especialmente no CEPA, e em grande parte sem a recompensa financeira que eu poderia ter obtido no meio corporativo. Mas esta é uma escolha que faço lucidamente, pois as minhas maiores recompensas são pessoais, e acima de tudo minha intenção é construir algo de que me orgulhe e que contribua para o futuro do meu filho e da nossa sociedade. Eu tenho conseguido seguir este caminho, bem ou mal, com a ajuda de todas as pessoas que encontrei nessa jornada. Mas ela não tem fim...

Criação e produção de objetos de aprendizagem, e pesquisa sobre eles. Esses são os três ingredientes que preciso. Atualmente tenho os dois primeiros, e já começo a lutar para conseguir o terceiro. É justamente onde este cargo de docente pode auxiliar. 

Independentemente disso, meu trabalho no CEPA continua, e para o próximo semestre (até o final do ano) tenho algumas metas bem claras:

\begin{compactitem}
	\item Finalizar a implantação do Scrum no CEPA;
	\item Buscar \foreign{feedback} dos alunos, tutores e educadores quanto aos Objetos de Aprendizagem disponibilizados;
	\item Difundir os Objetos de Aprendizagem entre os professores-autores, educadores e tutores (muitos deles sequer sabem que eles existem, e outros não imaginam que existe uma equipe capaz de realizar suas ideias);
	\item Difundir o padrão SCORM entre os professores autores, educadores e tutores;
	\item Submeter dois ou três trabalhos, em parceria com colegas acadêmicos do CEPA, para os próximos congressos da ABED (Associação Brasileira de Ensino a Distância), baseados em nossas experiências nos projetos Aulas Interativas, Univesp e Redefor.
\end{compactitem}

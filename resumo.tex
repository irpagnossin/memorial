\chapter{Dados pessoais e produções relevantes}
\label{cap:resumo}

%--------------------------------------------------------------------------
\section*{Dados pessoais e de contato}
\addcontentsline{toc}{section}{Dados pessoais e de contato}

\begin{compactdesc}
	\item[Nome:] Ivan Ramos Pagnossin
	\item[RG:] 15.420.406-7
	\item[CPF:] 179.905.018-13
	\item[Celular:] (11) 964.344.513
	\item[Endereço residencial:] Rua Milton Soares, 213, torre 2, apto. 55 --- São Paulo/SP --- CEP 05382-010
	\item[Telefone comercial:] (11) 3091-6695 ou 3091-6709
	\item[Endereço comercial:] CEPA (Centro de Ensino e Pesquisa Aplicada), Rua do Matão, Travessa R, 187
Edifício Van de Graaf --- Cidade Universitária-USP --- São Paulo/SP --- CEP 05508-090
	\item[e-mail:] ivan.pagnossin@gmail.com
\end{compactdesc}

\begin{tabbing}
Meus currículos na Internet: \=\url{http://goo.gl/lKz1u}\`Currículo Lattes\\
														 \>\url{http://goo.gl/jWcJD}\`LinkedIn\\
														 \>\url{http://goo.gl/N1MUC}\`curriculum.com.br
\end{tabbing}

%--------------------------------------------------------------------------
\section*{Formação acadêmica e titulação}
\addcontentsline{toc}{section}{Formação acadêmica e titulação}

\begin{compactdesc}
	\item[Ensino Fundamental:] Instituto São Pio X, de 1982 a 1990.
	\item[Ensino Médio:] Liceu de Artes e Ofícios de São Paulo, de 1991 a 1995.
	\item[Ensino Médio Profissionalizante:] Liceu de Artes e Ofícios de São Paulo, de 1991 a 1995. Título de Técnico em Eletrônica.
	\item[Graduação:] IFUSP, de 1997 a 2002. Título de Bacharel em Física Básica.
	\item[Iniciação científica:] Laboratório de Física de Plasmas do IFUSP, de 1997 a 1998.
	\item[Mestrado:] IFUSP, de 2002 a 2004. Título de Mestre em Ciências.
	\item[Doutorado:] IFUSP, de 2005 a 2007. Título de Doutor em Ciências.
\end{compactdesc}

%--------------------------------------------------------------------------
\section*{Formação complementar}
\addcontentsline{toc}{section}{Formação complementar}

\begin{compactitem}
	\item \textsl{Curso Profissional de Datilografia}, Tecla Escola de Datilografia, 1989.
	\item \textsl{Cursinho para o exame vestibulinho}, Centro Educacional Desafio, 1990.
	\item \textsl{Desenho técnico}, Liceu de Artes e Ofícios de São Paulo, 1991.
	\item \textsl{Desenho artístico}, Liceu de Artes e Ofícios de São Paulo, 1992.
	\item \textsl{Reconhecimento do céu}, Escola Municipal de Astrofísica Planetário Municipal, 1992 (15 horas).
	\item \textsl{Tópicos de Astronomia: movimentos da Terra}, Escola Municipal de Astrofísica Planetário Municipal, 1993 (10 horas).
	\item \textsl{Fundamentos de Astrofísica II --- Evolução estelar}, Escola Municipal de Astrofísica Planetário Municipal, 1993 (30 horas).
	\item \textsl{Cursinho para o exame vestibular}, Etapa, 1996.
	\item \textsl{Alemão}, Faculdade de Filosofia, Letras e Ciências Humanas, 2003 (ouvinte do primeiro semestre do curso de Letras).
	\item \textsl{Inglês}, Top English, 2002.
	\item \textsl{Curso de voo a vela}, Aeroclube Politécnico de Planadores, 2004--2009 (40 horas de voo).
	\item \textsl{Curso preparatório para o TOEFL}, Faculdade de Filosofia, Letras e Ciências Humanas da USP, 2006.
	\item \textsl{Francês}, Francês em casa, 2007 (50 horas).
	\item \textsl{Moodle na USP}, CEPA/CTI, 2008.
	\item \textsl{Desenvolvimento Web com HTML, CSS e JavaScript}, Caelum Ensino e Inovação, 2010 (20 horas).
	\item \textsl{Leader Training I}, Arita Treinamentos, 2010 (35 horas).
	\item \textsl{Gerenciamento ágil de projetos com Scrum}, Caelum Ensino e Inovação, 2010 (20 horas).
	\item \textsl{Brigadista para combate ao incêndio}, Centro de Treinamento da Rochácara Ecofire, 2010 (12 horas).
	\item \textsl{Leader Training II}, Arita Treinamentos, 2011 (35 horas).
	\item \textsl{Capacitação SMART Board}, SMART, 22 de julho de 2011 (8 horas).
\end{compactitem}

%--------------------------------------------------------------------------
\section*{Desenvolvimento de material didático ou instrucional}
\addcontentsline{toc}{section}{Desenvolvimento de material didático ou instrucional}

\begin{compactitem}
	\item 16 Objetos de Aprendizagem para \foreign{web} (Física), no âmbito do projeto TIDIA-Ae.
	\item 104 Objetos de Aprendizagem para lousa eletrônica (Matemática e Língua Portuguesa), no âmbito do projeto Aulas Interativas, sendo 31 produzidas por mim e 73 sob a minha coordenação.
	\item 118 Objetos de Aprendizagem para \foreign{web} (Física e Matemática), no âmbito dos projetos Univesp e Redefor, sendo 32 produzidas por mim e 86 sob a minha coordenação.
\end{compactitem}

%--------------------------------------------------------------------------
\section*{Produção bibliográfica}
\addcontentsline{toc}{section}{Produção bibliográfica}

\begin{compactitem}

	\item Dissertações e teses
	\begin{compactitem}
		\item \textsl{Propriedades de transporte elétrico de gases bidimensionais de elétrons nas proximidades de pontos-quânticos de InAs}, Dissertação de Mestrado, IFUSP (2004).
		\item \textsl{Pontos-quânticos: fotodetectores, localização-fraca e estados de borda contra-rotativos}, Tese de Doutorado, IFUSP (2008).
	\end{compactitem}
	
	\item Artigos completos publicados em periódicos
	\begin{compactitem}
		\item I. R. Pagnossin, E. C. F. da Silva, A. A. Quivy, S. Martini e C. S. Sergio, \textsl{The quantum mobility of a two-dimensional electron gas in selectively doped GaAs/InGaAs quantum wells with embedded quantum dots}, J. Appl. Phys. \textbf{97}, 113709 (2005).
		\item I. R. Pagnossin, A. K. Meikap, A. A. Quivy, G. M. Gusev, \textsl{Electron dephasing scattering rate in two-dimensional GaAs/InGaAs heterostructures with embedded InAs quantum dots}, J. Appl. Phys. \textbf{104}, 073723 (2008).
		\item I. R. Pagnossin, A. K. Meikap, T. E. Lamas, G. M. Gusev, J. C. Portal, \textsl{Anomalous dephasing scattering rate of two-dimensional electrons in double quantum well structures}, Phys. Rev. B, Condensed Matter and Materials Physics \textbf{78}, 115311 (2008). 
	\end{compactitem}
		
	\item Trabalhos completos publicados em anais de congressos
	\begin{compactitem}
		\item I. R. Pagnossin, C. C. Cavalcanti, R. T. Soledade, G. da C. Marques, \textsl{Objetos de aprendizagem interativos: análise do desempenho dos alunos de ciências}, no II Congresso Internacional TIC e Educação (ticEduca), Lisboa, Portugal, 2012.
		\item I. R. Pagnossin, E. C. F. da Silva, A. A. Quivy, S. Martini, C. S. Sergio, \textsl{The influence of strain fields around InAs quantum dots on the transport properties of a two-dimensional electron gas confined in GaAs/InGaAs wells}, no 12th Brazilian Workshop on Semiconductor Physics, 2005, São José dos Campos. Brazilian Journal of Physics, 2005.
		\item I. R. Pagnossin, E. C. F. da Silva, A. A. Quivy, S. Martini, C. S. Sergio, \textsl{Scattering processes on a quasi-two-dimensional electron gas in GaAs/InGaAs selectively doped quantum wells with embedded quantum dots}, no 12th Brazilian Workshop on Semiconductor Physics, 2005, São José dos Campos. Brazilian Journal of Physics, 2005.
		\item I. R. Pagnossin, G. M. Gusev, A. C. Seabra, A. A. Quivy, T. E. Lamas, J.-C. Portal, \textsl{Quantum Hall effect in bilayer system with array of antidots}, no 28th International Conference on the Physics of Semiconductors (ICPS-28), 2006, Viena. 28th International Conference on the Physics of Semiconductors (ICPS-28), 2006. p.~96.
		\item I. R. Pagnossin, G. M. Gusev, N. M. Sotomayor, A. C. Seabra, A. A. Quivy, T. E. Lamas, J.-C. Portal, \textsl{Quantum Hall effect in bilayer system with array of antidots}, no 28th International Conference on the Physics of Semiconductors (ICPS-28) --- Viena/Austria, 2006, Viena. Physics of Semiconductors, 28th International Conference, 2006. pp.~677--678.
	\end{compactitem}
	
	\item Resumos publicados em anais de congressos
	\begin{compactitem}
		\item I. R. Pagnossin, E. C. F. da Silva, A. A. Quivy, J. R. Leite, S. Martini, C. S. Sergio, \textsl{The influence of an InAs layer on the quantum mobility of a two-dimensoinal electron gas in GaAs/InGaAs selectively doped quantum wells}, no XXVII Encontro Nacional da Matéria Condensada, 2004, Poços de Caldas. XXVII ENFMC, 2004. v. 1. pp.~399--399.
		\item I. R. Pagnossin, A. K. Meikap, A. A. Quivy, G. M. Gusev, \textsl{Weak localization and interaction effects in GaAs/InGaAs heterostructures with nearby InAs quantum-dots}, no 13th Brazilian Workshop on Semiconductor Physics --- São Paulo/SP, 2007, São Paulo. 13th Brazilian Workshop on Semiconductor Physics --- São Paulo/SP, 2007.
		\item I. R. Pagnossin, G. M. Gusev, A. C. Seabra, A. A. Quivy, T. E. Lamas, J.-C. Portal, \textsl{Quantum Hall effect in bilayer system with array of antidots}, no 28th International Conference on the Physics of Semiconductors (ICPS-28) --- Viena/Austria, 2006, Viena. 28th International Conference on the Physics of Semiconductors (ICPS-28) --- Viena/Austria, 2006.	
	\end{compactitem}
	
	\item Apresentações de trabalhos
	\begin{compactitem}
		\item R. S. Morais, V. C. R. Sarnighausen, I. R. Pagnossin, R. N. Marques, \textsl{O curso LC-EaD da USP --- Um olhar para o pólo Piracicaba no componente curricular Fundamentos de Matemática}, XVI Conferência GPIMEM: tecnologias digitais em Educação Matemática, Rio Claro (SP), Brasil (2013).
		\item I. R. Pagnossin, P. P. L. Oliveira, M. H. Klein, \textsl{Inovação e Qualidade: Soluções Integradas na Produção de Materiais para EaD} (mesa redonda), I Jornada de Design Instrucional (JORDI), São Paulo (SP), 2012.
		\item I. R. Pagnossin, G. da C. Marques, M. H. Klein, \textsl{Educação a distância na USP: da concepção à realização} (mesa redonda), 17\textordmasculine\ Congresso Internacional ABED de Educação a Distância, Manaus (AM), Brasil (2011).
		\item I. R. Pagnossin, C. C. Cavalcanti, R. T. Soledade, G. da C. Marques, \textsl{Participação e desempenho de alunos no uso de objetos de aprendizagem interativos que simulam situações-problema na licenciatura em ciências da USP/Univesp}, I Simpósio Internacional de Educação a Distância e I Encontro de Pesquisadores em Educação a Distância, São Carlos (SP), Brasil (2012).
		\item I. R. Pagnossin, M. H. Klein, P. P. L. Oliveira, \textsl{Inovação e Qualidade: Soluções Integradas na Produção de Materiais para EaD} (mesa redonda), 18\textordmasculine\ Congresso Internacional ABED de Educação a Distância, São Luís (MA), Brasil (2012).
	\end{compactitem}
\end{compactitem}

%--------------------------------------------------------------------------
\section*{Relatórios técnicos}
\addcontentsline{toc}{section}{Relatórios técnicos}

\begin{compactitem}
	\item 12 relatórios de atividade, para a FAPESP, referentes ao trabalho de desenvolvimento de \foreign{softwares} educacionais utilizando Java, no âmbito do projeto TIDIA-Ae, durante o período de dezembro de 2007 a janeiro de 2009. Esses relatórios podem ser obtidos através de \cite{irpagnossin-stoa}.
	\item 7 relatórios mensais de atividade, para o Instituto de Pesquisa Eldorado, referentes ao trabalho de produção de \foreign{softwares} educacionais para o projeto Aulas Interativas, durante o período de dezembro de 2009 a junho de 2010.
\end{compactitem}

%--------------------------------------------------------------------------
\section*{Bolsas e auxílios}
\addcontentsline{toc}{section}{Bolsas e auxílios}

\begin{compactitem}
	\item Ensino Médio Profissionalizante, Liceu de Artes e Ofícios de São Paulo, 1991--1995.
	\item Iniciação científica, CNPq, 1997.
	\item Iniciação científica, CNPq, 1998.
	\item Mestrado, FAPESP, 2002--2004.
	\item Doutorado, FAPESP, 2005--2007.
	\item Capacitação Técnica, nível TT-4, FAPESP, 2008.
	\item Capacitação Técnica, nível TT-5, FUSP, 2010.
\end{compactitem}

%--------------------------------------------------------------------------
\section*{Auxílios para viagens ao exterior}
\addcontentsline{toc}{section}{Auxílios para viagens ao exterior}

\begin{compactitem}
	\item Grenoble High Magnetic Field Laboratory, Grenoble, França (março a maio de 2007). Estágio financiado pela FAPESP como parte do trabalho de doutorado.
\end{compactitem}


%--------------------------------------------------------------------------
\section*{Participação em eventos}
\addcontentsline{toc}{section}{Participação em eventos}

\begin{compactitem}
	\item XVII Encontro Nacional de Física da Matéria Condensada, Poços de Caldas (MG), Brasil (2004).
	\item 12th Brazilian Workshop on Semiconductor Physics, São José dos Campos (SP), Brasil (2005).
	\item XVIII Encontro Nacional de Física da Matéria Condensada, Santos (SP), Brasil (2005).
	\item 17\textordmasculine\ Congresso Internacional ABED de Educação a Distância (CIAED), Manaus (AM), Brasil (2011).
	\item I Simpósio Internacional de Educação a Distância e I Encontro de Pesquisadores em Educação a Distância (SIED:EnPED), São Carlos (SP), Brasil (2012).
	\item 18\textordmasculine\ Congresso Internacional ABED de Educação a Distância (CIAED), São Luís (MA), Brasil (2012).
\end{compactitem}

%--------------------------------------------------------------------------
\section*{Organização de eventos}
\addcontentsline{toc}{section}{Organização de eventos}

\begin{compactitem}
	\item \LaTeX\ e a Internet, CEPA (2008).
\end{compactitem}

%--------------------------------------------------------------------------
\section*{Projetos de pesquisa}
\addcontentsline{toc}{section}{Projetos de pesquisa}

\begin{compactitem}
	\item Estudo e desenvolvimento dos processos de criação e produção de Objetos de Aprendizagem para o projeto Univesp, junto à CTI e financiado pela FUSP (projeto Univesp), 2010.
\end{compactitem}

%--------------------------------------------------------------------------
\section*{Cursos e treinamentos produzidos}
\addcontentsline{toc}{section}{Cursos e treinamentos produzidos}

\begin{compactitem}
	\item \textsl{Usando \LaTeX; pensando \TeX}, curso semi-presencial com efoque prático (parte não-presencial através do sistema de gerenciamento de cursos Moodle da CTI, domínio \url{moodle.stoa.usp.br}).
	\item \textsl{Aula temática com lousa interativa}, produzida para a abertura oficial do projeto Aulas Interativas, na Secretaria de Estado da Educação, em 6 de novembro de 2009.
	\item \textsl{Processo de Criação de Objetos de Aprendizagem --- Usos do Padrão SCORM}, disciplina do curso de pós-graduação a distância em \foreign{Design} Instrucional do Senac/SP.
\end{compactitem}

%--------------------------------------------------------------------------
\section*{Cursos e treinamentos ministrados}
\addcontentsline{toc}{section}{Cursos e treinamentos ministrados}

\begin{compactitem}
	\item \textsl{Usando \LaTeX; pensando \TeX}, CEPA, 15 horas (2008).
	\item \textsl{Usando \LaTeX; pensando \TeX}, CEPA/CTI, 20 horas (2008).
	\item \textsl{Usando \LaTeX; pensando \TeX}, CEPA/CTI, 24 horas (2009).
	\item Treinamento de professores da rede pública de Hortolândia para a aula temática apresentada na abertura oficial do projeto Aulas Interativas, Diretoria de Ensino de Sumaré (3--5 de novembro de 2009).
	\item \textsl{Processo de Criação de Objetos de Aprendizagem --- Usos do Padrão SCORM}, disciplina do curso de pós-graduação a distância em \foreign{Design} Instrucional do Senac/SP (3--28 de agosto de 2012).
	\item \textsl{Processo de Criação de Objetos de Aprendizagem --- Usos do Padrão SCORM}, disciplina do curso de pós-graduação a distância em \foreign{Design} Instrucional do Senac/SP (março de 2012).
\end{compactitem}

%--------------------------------------------------------------------------
\section*{\foreign{Softwares} produzidos}
\addcontentsline{toc}{section}{\foreign{Softwares} produzidos}

\begin{compactitem}
	\item Script LabTalk para a obtenção das mobilidades quânticas das sub-bandas de gases bidimensionais de elétrons a partir de oscilações de magnetoresistência (efeito Shubnikov-de Haas).
\end{compactitem}

%--------------------------------------------------------------------------
\section*{Palestras}
\addcontentsline{toc}{section}{Palestras}

\begin{compactitem}
	\item \textsl{Um modelo de aula interativa com lousa eletrônica}, CEPA, 19 de maio de 2009 (apresentação da proposta do CEPA para o projeto Aulas Interativas, para membros da SEE e da Dell).
	\item \textsl{Um modelo de aula interativa com lousa eletrônica}, Diretoria de Ensino de Sumaré, julho de 2009 (apresentação do projeto Aulas Interativas para os diretores das escolas públicas de Hortolândia).
	\item \textsl{Tecnologia educacional: atividades interativas para EaD e lousas digitais}, CEPA, 23 de outubro de 2009 [apresentação feita para os alunos da disciplina FAP0459 (tecnologia educacional), a convite do Prof. Dr. Ewout ter Haar].
	\item \textsl{Atividades Interativas para EaD e lousas eletrônicas}, CTI, 13 de julho de 2010 (apresentação feita para os coordenadores do projeto Univesp, sobre uma proposta de atividades interativas para o mesmo projeto).
\end{compactitem}

%--------------------------------------------------------------------------
\section*{Orientações e supervisões concluídas}
\addcontentsline{toc}{section}{Orientações e supervisões concluídas}

\begin{compactitem}
	\item Coordenação de uma equipe de 4--6 programadores para a produção de \foreign{softwares} educacionais para o projeto Aulas Interativas. De julho de 2009 a agosto de 2010.
	\item Coordenação de uma equipe de 6 programadores e 1 ilustrador para a produção de Objetos de Aprendizagem para os projetos Univesp e Redefor. De agosto de 2010 a dezembro de 2011.
\end{compactitem}

%--------------------------------------------------------------------------
%\section*{Orientações e supervisões em andamento}
%\addcontentsline{toc}{section}{Orientações e supervisões em andamento}

%\begin{compactitem}
	
%\end{compactitem}



\chapter{Formação acadêmica}

Minha formação acadêmica sempre foi voltada para o estudo da Física Básica e Aplicada, particularmente na área de materiais semicondutores e de fenômenos de transporte eletrônico. No entanto, ela sofreu grande influência da minha formação profissional (capítulo~\ref{cap:formacao-profissional}) --- bem como a influenciou ---, especialmente no que concerne o desenvolvimento de \foreign{software}.

\section{Graduação e iniciação científica}
\label{sec:graduacao}

Minha formação acadêmica começou em fevereiro de 1997, quando fui aprovado no exame vestibular da USP para o curso de Bacharelado em Física, no \foreign{campus} da capital. A escolha pela Física foi moldada, anos antes, pelo contato com circuitos eletrônicos, no ensino médio profissionalizante (seção~\ref{sec:liceu}), bem como por uma paixão ainda mais antiga: a aviação. Já a opção pelo bacharelado foi feita pelo desejo de aprender \emph{profundamente} fenômenos físicos e técnicas matemáticas, em muito alimentado pelo aprendizado autodidata de cálculo diferencial e integral, ainda no ensino médio.

Na verdade, o fato de eu já conhecer o cálculo diferencial e integral ao começar a graduação ajudou-me enormemente, permitindo-me aproveitar melhor os conceitos e técnicas ensinados, bem como ir além deles, em todas as disciplinas cursadas. De fato, em oposição ao que ocorreu durante o ensino médio (veja a seção \ref{sec:liceu}), eu me destaquei em praticamente todas as disciplinas, chegando a atingir médias como 9,5 em disciplinas como Cálculo, Álgebra Linear, Física Atômica e Molecular, entre outras.

O ensino médio profissionalizante teve outra clara e positiva influência na minha graduação: logo na primeira semana consegui uma bolsa de iniciação científica do CNPq no Laboratório de Física de Plasmas, orientado pelo Prof. Dr. Ivan Cunha Nascimento, e em muito auxiliado pelo funcionário Juan Iraburu Elizondo. A proposta era a de estudar a chamada curva de \foreign{breakdown}, que caracterizava a formação de plasma no Tokamak. Os resultados deste trabalho foram apresentados no Simpósio de Iniciação Científica da USP de 1998. Permaneci na iniciação científica até meados daquele ano, quando então fui contratado pela Caixa Econômica Federal (CEF). No entanto, exceto pelas grandes e benéficas influências intelectuais que sofri no laboratório, a iniciação científica pouco contribuiu para a minha carreira.

Mais importantes talvez tenham sido as disciplinas de introdução à computação e de cálculo numérico, que contribuiram para um novo rumo na minha formação profissional, dali alguns anos (seção \ref{sec:eletropiezo}), e que sigo até hoje. Igualmente importante foi a disciplina de introdução à Física do Estado Sólido, na qual conheci a \profa\ Dr\rlap{\textordfeminine}. Euzi Conceição Fernandes da Silva, que me convidou para fazer a pós-graduação no Departamento de Física dos Materiais (DFMT) do IF da USP (IFUSP) e que me \emph{muito} auxiliou desde então.

Finalmente, outra grande conquista pessoal ocorrida durante a graduação foi a auto-instrução da língua inglesa. Embora eu tenha feito alguns cursos esporadicamente (nem mencionados no capítulo~\ref{cap:resumo}), foi na graduação que adquiri maturidade intelectual para assimilar a língua, principalmente através dos livros e filmes da videoteca do IFUSP.

Assim, concluí a graduação em 2001 com aproveitamento médio de 83\% e com habilitação em Física Básica. Meu intuito era obter também a habilitação em microeletrônica, mas isto estenderia a graduação por pelo menos mais um ano, o que eu não estava disposto a aceitar, pois já havia gasto um ano extra no ensino médio profissionalizante, outro no cursinho e mais outro na graduação (quando migrei do período matutino para o noturno, na ocasião de minha contratação pela CEF). Ademais, a oferta de pós-graduação com a \profa\ Euzi já me levava para a área do transporte eletrônico. Deste modo, desisti da habilitação em microeletrônica.
 
\section{Mestrado}

Na época em que me candidatei ao mestrado \foreign{stricto sensu}, em 2001, logo após concluir a graduação, a FAPESP já iniciava seu movimento em prol do doutoramento direto; sem mestrado. No entanto, apesar do meu desespero em avançar na carreira acadêmica e por orientação da \profa\ Euzi, decidi pelo caminho mais longo: o mestrado, na certeza de que era um passo importante que não deveria ser pulado (ainda hoje acredito que esta foi uma escolha acertada).

Minha opção por uma pós-graduação experimental é outra que merece explicação: ao longo da graduação eu percebi que tinha muita facilidade com a teoria, mas nem tanto com a prática. Assim, minha expectativa era que um mestrado experimental me permitiria corrigir este desequilíbrio. Isto realmente aconteceu, mas a minha ``veia teórica'' sempre deu suas contribuições, no mestrado e no doutorado, e ainda hoje é mais expressiva.

A proposta para o mestrado era caracterizar a evolução de pontos-quânticos auto-organizados através de medidas ópticas (fotoluminescência) e de transporte eletrônico (efeitos Hall quântico inteiro e Shubnikov-de Haas) em baixas temperaturas ($\sim\unit{1,4}{\kelvin}$). E deste modo aprendi a manusear nitrogênio e hélio-4 líquidos, bem como equipamentos complexos como criostatos, bombas de vácuo, amplificadores \foreign{lock-in}, espectrômetros, \foreign{lasers} de alta potência \etc. Aprendi também técnicas como litografia, microscopias de varredura (principalmente de força atômica), crescimento epitaxial molecular e confecção de contatos eletrônicos por difusão. Em suma, o mestrado foi um período de intenso aprendizado, como deveria ser.

Como resultado deste trabalho, chegamos à conclusão de que a tensão mecânica acumulada nos pontos-quânticos, por consequência do crescimento epitaxial, afeta as mobilidades dos elétrons. Este foi um resultado inédito na literatura científica (até onde sabemos), o que nos rendeu um artigo \cite{pagnossin-2005-1}, uma exposição dele (pôster) no XVII Encontro Nacional de Física da Matéria Condensada (ENFMC), em 2004 \cite{pagnossin-2004-2}, e, mais tarde, uma versão expandida dele no XVIII ENFMC e no \foreign{12th Brazilian Workshop on Semiconductor Physics} (BWSP), em 2005 \cite{pagnossin-2005-2, pagnossin-2005-3}. Além disso, este era um resultado importante para o rumo que nosso grupo de pesquisa buscava naquela época: o estudo e confecção de \foreign{lasers} e detectores de infra-vermelho baseados em pontos-quânticos.

Além desses resultados, dois outros destacaram-se: o primeiro foi a dedução matemática da técnica utilizada na análise das oscilações de magnetoresistência (o chamado efeito Shubnikov-de Haas), que aparentemente perdeu-se na literatura (nós nunca a encontramos). Esta dedução está registrada nos apêndices da minha dissertação de mestrado \cite{pagnossin-2004-1}. O segundo foi o desenvolvimento de um script\footnote{Escrito em LabTalk, linguagem de script do \foreign{software} de análise de dados Microcal Origin.} para automatizar parte dessa análise, o que permitiu reduzir o tempo dela em aproximadamente 90\%. Este script, mais a compreensão do método adquirida na dedução matemática dele, permitiu-me desenvolver uma pesquisa informal paralela, e estabelecer os limites da técnica e seus efeitos sobre os dados.

Assim, concluí o mestrado em maio de 2004 com resultados empolgantes (algo incomum de acontecer, segundo a \profa\ Euzi), e os apresentei para a banca examinadora, composta pela \profa\ \dra\ Lucy Vitoria Credidio Assali (IFUSP), pelo Prof. Dr. Marcelo Nelson Paez Carreño (da Escola Politécnica da USP) e, claro, pela \profa\ \dra\ Euzi Conceição Fernandes da Silva.

\section{Doutorado}

A proposta de trabalho para o doutorado era caracterizar as possíveis heteroestruturas-base de detectores de infra-vermelho baseados em pontos-quânticos. A discussão, na literatura científica, sobre qual seria a melhor estrutura para este dispositivo, estava no auge. Além disso, nosso grupo de pesquisa havia conseguido um resultado até então dado como impossível: a absorção, por pontos-quânticos, de ondas de infra-vermelho com comprimento de onda de \unit{1,5}{\micro\metre} \cite{silva-2003}. A importância deste resultado, e de todos os estudos que se seguiram, residia no fato de que a fibra óptica utilizada em telecomunicações apresenta um mínimo absoluto de absorção nesta frequência, de modo que dispositivos operando nesta faixa trariam grandes benefícios econômicos.

O estudo começou, então, por experimentar algumas possíveis configurações de heteroestruturas, conforme propostas existentes na literatura científica. A ideia era utilizar nossa já conhecida caracterização eletrônica para determinar as mobilidades dos elétrons e, com isso, identificar a melhor configuração para o dispositivo.

No entanto, aproximadamente um ano após o início do doutorado, a \profa\ Euzi foi para o \foreign{Center for Quantum Devices}, nos EUA, a convite da \profa\ Manijeh Razeghi, e desta maneira fui obrigado a mudar de orientador.

O Prof. Dr. Guennadii Michailovich Gusev, que assumiu a chefia do DFMT com a morte do Prof. Dr. José Roberto Leite, em 2004, cordialmente aceitou orientar-me a partir daí. No entanto, sua linha de pesquisa concentrava-se em fenômenos de Física Básica, como o efeito Hall quântico fracionário, transporte eletrônico em sistemas mesoscópicos, efeitos de \foreign{spin} em sistemas bidimensionais \etc. E deste modo minha pesquisa foi alterada para o estudo de redes de anti-pontos-quânticos.

Esta mudança foi muito benéfica, pois a visão do Prof. Gusev sobre os assuntos da pesquisa era deveras diferente daquele da \profa\ Euzi, de modo que isto me deu perspectivas novas. Ademais, aprendi inúmeras outras técnicas experimentais, como manusear hélio-3, nanolitografia por microscopia eletrônica, confecção de \foreign{gates} de ouro por evaporação, além de formalismos matemáticos como o de Landauer Büttiker, entre outros. No entanto, a troca de orientador teve um efeito severo sobre minha pesquisa: eu praticamente a desenvolvi sozinho. Embora o Prof. Gusev sempre se dispusesse a discutir qualquer assunto, sua presença na minha pesquisa não era tão evidente quanto a da \profa\ Euzi. Isto prejudicou um pouco a qualidade do trabalho que desenvolvi, mas também me tornou mais independente.

Durante o desenvolvimento desse trabalho, encontramos, por acaso, evidências experimentais dos chamados estados de borda contra-rotativos, previstos teoricamente em 1992 \cite{johnson-1992}, mas até então não observados. E a partir daí minha pesquisa voltou-se para este assunto.

Entretanto, a construção das amostras requeridas para este estudo estava no limiar da capacidade técnica que tínhamos à disposição, o microscópio eletrônico do Laboratório de Sistemas Integráveis (LSI) da Escola Politécnica. Não obstante isso, nosso acesso a este equipamento era raro, o que tornava deveras demorado obter um conjunto de amostras. Adicione a isto a constante dificuldade em conseguir hélio-4 para os criostatos (a demanda do grupo era grande, em parte devido ao Detector de Ondas Gravitacionais Mário Schenberg, que se preparava para entrar em operação), e o resultado é que nunca conseguimos reproduzir aqueles resultados.

Apesar disso, conseguimos apresentar as primeiras evidências no \foreign{28th International Conference on the Physics of Semiconductors}, o mais importante congresso de Física de Semicondutores, em 2006, na Áustria \cite{pagnossin-2006}.

Passados dois anos de doutorado eu estava com um grande problema nas mãos: minha pesquisa inicial, sobre fotodetectores, havia sido interrompida prematuramente, e aquela sobre os estados de borda contra-rotativos não avançava o suficiente para apresentar uma tese de doutorado. Foi então que procurei o Prof. Dr. Ajit Kumar Meikap, do \foreign{National Institute of Technology}, na Índia (que estava passando uma temporada no Brasil, a convite do Prof. Gusev), e propus que fizéssemos estudos de localização-fraca em amostras mais simples (poços-quânticos duplos e parabólicos), dentre elas aquelas utilizadas no meu mestrado (pontos-quânticos).

O Prof. Meikap havia desenvolvido, durante sua estada no Brasil, todo o ferramental para analisar dados conforme os mais recentes estudos sobre localização-fraca, mas não tinha o que analisar. Eu, por outro lado, tinha um enorme conjunto de medidas já prontas, e inúmeras outras que podiam ser feitas com facilidade, pois na época eu estava fazendo um estágio de dois meses e meio no \foreign{Grenoble High Magnetic Field Laboratory}, na França, sob supervisão do Prof. Dr. Jean-Claude Portal, com equipamentos à minha disposição quase exclusiva.

Esta parceria rendeu dois artigos \cite{pagnossin-2008-1, pagnossin-2008-2} e uma exposição no \foreign{13th} BWSP, em 2007 \cite{pagnossin-2007}.

Na tese de doutorado apresentei, então, três conjuntos de resultados: aqueles dos fotodetectores (embora incompletos, já era possível tirar algumas conclusões que guiassem a confecção de fotodetectores baseados em pontos-quânticos), aqueles dos estados de borda contra-rotativos (os que eu mais gostei, apesar de tudo) e aqueles relacionados às medidas de localização-fraca.

A defesa da tese de doutoramento ocorreu em abril de 2008, tendo como banca examinadora o Prof. Dr. Antônio Carlos Seabra (EPUSP), o Prof. Dr. Eliermes Arraes Meneses (UNICAMP), a \profa\ \dra\ Euzi Conceição Fernandes da Silva (IFUSP), o Prof. Dr. Fernando Iikawa (UNICAMP) e a \profa\ \dra\ Lucy Vitória Credidio Assali (IFUSP).

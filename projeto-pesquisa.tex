\chapter{Projeto de pesquisa}
\label{cap:projeto-pesquisa}

\section{Introdução}

O processo de ensino-aprendizagem assistido por computador (TICE, de Tecnologia da Informação e Comunicação para a Educação), como laboratórios de informática, cursos a distância \cite{mit, redefor}, lousas eletrônicas \cite{aulas-interativas} e dispositivos móveis \cite{ipad}, entre outros, tem se tornado cada vez mais presente como uma complementação de qualidade à educação tradicional. E particularmente no caso do Ensino a Distância (EaD), constitui também numa alternativa mais econômica e acessível, capaz de transpor barreiras geográficas e integrar diferentes culturas, de modo a realizar o que hoje se conhece por aprendizado colaborativo.

Devido a este sucesso das TICE, a demanda por \foreign{softwares} desenvolvidos para este fim também ganha espaço. Exemplos já bastante difundidos são as comunidades virtuais (Facebook, Orkut, Stoa \etc), \foreign{blogs}, \foreign{chats}, fóruns \etc. E dentre elas há um subconjunto de \foreign{softwares} educativos que exploram a capacidade de cálculo do computador para facilitar ou otimizar a compreensão de conceitos e técnicas \cite{melare-2009}. Por exemplo, utilizando-se dicionários digitais é possível explorar dinamicamente a ocorrência de prefixos e sufixos nos vocábulos da língua portuguesa. Ou ainda, pode-se explorar os efeitos de se ignorar a aproximação de ângulos pequenos na modelagem do pêndulo simples. De fato, algumas iniciativas já estão disponíveis, como o Projeto Homem Virtual \cite{homem-virtual} e \foreign{softwares} de geometria dinâmica \cite{geogebra, igeom}.

Outro exemplo de grande sucesso e muito difundido são os simuladores de voo: \foreign{softwares} como o Flight Simulator e o X-Plane permitem que o candidato a piloto aprenda muitos conceitos antes mesmo de entrar num avião. Realmente, nas normas da ANAC (Agência Nacional de Aviação Civil) consta que ``30 horas de voo podem ser substituídas por 30 horas de instrução em simulador'' (RBHA 61, para a habilitação em voo por instrumentos). E com relação a isso eu posso dar um testemunho: eu, como piloto de planador, aprendi algumas das técnicas que uso em voo (real) num simulador chamado Condor Soaring.

Assim, este tipo de recurso pode ser utilizado desde o ensino básico \cite{coelho-sabido} até o ensino superior \cite{phet}, e mesmo em pós-graduação e cursos técnicos \cite{lock-in}.

Mas apesar de todo este potencial, a indústria deste tipo de material ainda não existe de fato. Ao invés disso, tudo que se tem são iniciativas isoladas, a maioria amadoras \cite{falstad, fendt}. Apenas recentemente algumas universidades no exterior começaram a se dedicar à produção e estudo desses materiais \cite{phet, merlot}, e algumas empresas (também no exterior) começam a enxergar mercado para eles. É onde esta proposta se encaixa, particularmente (mas não exclusivamente) no âmbito da Universidade Virtual do Estado de São Paulo (Univesp).

\section{Justificativa}

A criação desses \foreign{softwares} educacionais [ou, mais ambiciosamente, Objetos de Aprendizagem (OA)] exige uma interdisciplinaridade ainda escassa hoje em dia, entre a área do saber a ser ensinado, a pedagogia e a área tecnológica. Como consequência, o processo de produção de OA é ainda pouco desenvolvido. Pelo lado tecnológico, a indústria que mais se aproxima dele é a de jogos eletrônicos; pelo lado pedagógico, o que mais se aproxima é o \foreign{designer} instrucional, profissão ainda pouco difundida no Brasil.

Existem ferramentas capazes de auxiliar educadores na criação de conteúdos mais interessantes que aqueles normalmente utilizados (textos, imagens e vídeos), como o eLML e o Adobe Captivate, mas em geral esses conteúdos não usufruem das vantagens do computador. Por exemplo, é comum ver questões criadas com ferramentas como estas que apenas reproduzem, no computador, questões de sala de aula. Algumas até utilizam efeitos especiais de transição, mas geralmente sem qualquer utilidade pedagógica. São, por assim dizer, recursos para agradar aos olhos. Certamente eles têm seus méritos, mas é justamente em seus defeitos que esta proposta se concentra.

\section{Objetivo}

Deste modo, {\bfseries o objetivo fundamental desta proposta é desenvolver e difundir conhecimento na criação \emph{e} produção\footnote{A \emph{criação} é a fase em que o OA é identificado e aperfeiçoado de modo a cumprir um objetivo pedagógico. A \emph{produção} é a etapa de realização da ideia surgida na criação.} de Objetos de Aprendizagem, e deste modo, estabelecer um caminho de formação para futuros profissionais nesta área}.

Por outro lado, \emph{não} é objetivo desta proposta criar ferramentas que facilitem a criação de Objetos de Aprendizagem por \emph{educadores}, a exemplo do Adobe Captivate. Ao invés disso, o intuito é permitir a formação de pessoas com vocação tanto para as áreas do saber a ser ensinado quanto para a tecnológica, supondo a existência do suporte pedagógico de um profissional qualificado para tal, como o \foreign{designer} instrucional.

Convém enfatizar que o projeto Aulas Interativas (seção~\ref{sec:aulas-interativas}), financiado pela Dell e em parceria com a SEE, evidencia o interesse comercial dessa empresa neste mercado, indicando uma possível tendência econômica e, por conseguinte, disponibilidade de postos de trabalho para profissionais qualificados na criação e produção de Objetos de Aprendizagem. É, portanto, mais uma justificativa para o desenvolvimento deste projeto.

\section{Proposta imediata}

Neste momento, o curso de Licenciatura em Ciências pela Univesp está em fase de criação e oferecimento, tornando-o a plataforma ideal para começar este trabalho. Por esta forma, a proposta \emph{imediata} deste trabalho é, nesta ordem:

\begin{compactenum}
	\item Criar e produzir Objetos de Aprendizagem para as disciplinas do curso;
	\item Oferecer esses OA para os alunos, como parte das atividades \foreign{online} (essas atividades, conforme o coordenador executivo do projeto Univesp, Prof. Dr. Gil da Costa Marques, são os recursos centrais do ambiente virtual de aprendizagem, \ie, o Moodle);
	\item Entrevistar os alunos, tutores e educadores acerca dos prós e contras desses OA, bem como obter sugestões de melhoria (dados úteis podem ser obtidos também através das avaliações automáticas do Moodle e de seus registros de log);
	\item Aperfeiçoar os OA produzidos.
\end{compactenum}

Este processo de pesquisa e produção é similar àquele utilizado no projeto PhET, da Universidade do Colorado \cite{phet}, e junto com sua extensa lista de estudos já efetuados \cite{phet-research}, constitui o modelo a ser seguido nas etapas iniciais.

Os passos 1 e 2 acima já são executados pelo CEPA atualmente, para qualquer disciplina dos projetos Univesp e Redefor. No entanto, o embasamento pedagógico ainda não está presente na fase de criação. Este é um erro a ser corrigido com a presença de um \foreign{designer} instrucional (primeira opção) ou com treinamento do profissional de criação de Objetos de Aprendizagem (segunda opção).

Essas ações corretivas são importantes, mas num primeiro momento a pesquisa propriamente dita pode se concentrar nos passos 3 e 4. E apenas para as disciplinas de Física e Matemática, áreas nas quais sou capacitado.

\section{Metodologia}

Essencialmente, a pesquisa nesta etapa inicial consistirá de entrevistas, formais ou informais, com todos aqueles que parcipam do processo de ensino e aprendizagem no dia-a-dia do curso da Univesp. São eles: os educadores, os tutores e, principalmente, os alunos (ou ``aprendizes'', como se diz no meio do EaD). Além disso, os próprios Objetos de Aprendizagem, quando confeccionados de modo a registrar o aproveitamento dos alunos (por exemplo, usando SCORM), podem oferecer informações importantes sobre sua eficácia. E mesmo os registros do próprio Moodle pode guardar informações úteis, como o número de tentativas de um mesmo OA ou o tempo gasto nele.

A expectativa é que a análise desses dados forneça direcionamento na criação e produção de novos Objetos de Aprendizagem e daqueles que originaram esses dados. E tendo isto em mente, torna-se imprescindível mencionar que as etapas 1--4 da seção anterior ocorrem para \emph{cada} OA. Em outras palavras: nós não vamos esperara acabar um semestre para, então, verificar o que deu certo e o que não; isto será feito rotineiramente.

\section{Propostas para o futuro}

A longo prazo, outras propostas de trabalho e pesquisa já podem ser cogitadas, embora a realização delas dependa do andamento da proposta imediata (seção anterior), bem como do desenvolvimento das TICE nos próximos anos. São elas, em ordem crescente de complexidade:

\begin{compactitem}
	\item Pesquisa, criação e produção de OA para lousas eletrônicas. O CEPA já detém algum conhecimento sobre este assunto, desenvolvido no projeto Aulas Interativas (seção~\ref{sec:aulas-interativas}), de modo que é bastante simples continuar este trabalho.
	\item Pesquisa, criação e produção de OA para dispositivos móveis, como celulares e \foreign{tablets}. O CEPA comprou recentemente um \foreign{tablet} baseado no sistema Android, e já começa a desenvolver conhecimento nessa linha.
	\item Incorporação de princípios de internacionalização e acessibilidade nos OA, o que ampliaria o público-alvo dos Objetos de Aprendizagem e beneficiaria pessoas com necessidades especiais. Além disso, um trabalho de pesquisa sobre como aplicar princípios de neurolinguística nos OA potencialmente os tornariam mais eficazes.
	\item Integração ente OA e vídeos, compondo um ``laboratório didático virtual''. A ideia é disponibilizar para o aluno equipamentos virtuais e vídeos pré-gravados que o orientam em cada tomada de decisão, certa ou errada. O aluno poderia inclusive chegar a ``destruir'' o aparelho se, por exemplo, permitir que muita luz atinja um fotodetector sensível.
	\item Extensão da pesquisa para outras áreas. A pesquisa proposta na seção anterior limita-se à Física e à Matemática apenas por que esta é a minha formação. No entanto, à medida que o projeto tomasse proporções maiores, parcerias poderiam ser feitas e pesquisas similares, desenvolvidas em outras áreas.
\end{compactitem}

Cada um dos itens acima, além de compreender um produto a ser desenvolvido, requer pesquisa similar àquela da seção anterior. Outros itens, como os abaixo, são contribuições que eu e uma equipe de criação de Objetos de Aprendizagem podemos realizar.

\begin{compactitem}
	\item Retomada do curso ``Usando \LaTeX; pensando em \TeX'' (seção~\ref{sec:latex}). Atualmente este curso encontra-se disponível nos servidores da CTI, mas o curso é oferecido de modo assíncrono e individual. No entanto, a demanda por esse conhecimento na USP e em outros centros de pesquisa justifica sua reativação, com a dedicação de tutores \foreign{online} aos alunos.
	\item Comunidade social de usuários de OA, integrada ao repositório de OA que já estamos desenvolvendo, e que permitiria a troca de ideias e experiências entre usuários de OA, bem como valioso \foreign{feedback} para a equipe de criação e produção deles. Mais do que isso, essas informações podem definir as necessidades mais imediatas dos educadores, de modo a guiar a criação e produção de OA.
	\item \foreign{Framework} para a criação de OA baseado em tecnologias e padrões abertos. A produção de Objetos de Aprendizagem requer a participação de programadores e ilustradores (pelo menos), e no que concerne a integração dos esforços de cada um desses profissionais, a plataforma Flash é hoje a melhor opção. Existem ferramentas livres que substituem o Flash na parte de programação, assim como existem ferramentas livres que substituem o Flash na parte de arte. Mas essas ferramentas não produzem resultados facilmente integráveis.
	\item Cursos de capacitação em programação para OA, desde paradigmas de programação (procedural, estruturada, orientada a objetos, protótipos, lógica e funcional) e técnicas (\foreign{design patterns}), passando por métodos numéricos (como derivação e integração numérica, quaternions \etc) até chegar nas diversas tecnologias disponíveis para a produção de OA, para \foreign{desktop}, Internet e dispositivos móveis.
	\item Estabelecer o CEPA como centro de criação de materiais didáticos para qualquer faculdade da USP que necessite deles. Este é um trabalho mais político que técnico, e por isso ainda bastante incerto.
\end{compactitem}

\section{Conclusão}

As TICE têm se mostrado um caminho sem retorno (as evidências aparecem todos os dias). E os ``\foreign{softwares} educacionais/educativos'' ou ``Objetos de Aprendizagem'' ou ``Unidades de Aprendizagem'' (...) são parte desta evolução. Algumas universidades e empresas no exterior já perceberam isso, e essa tendência começa a afetar o Brasil: tome como exemplo a SEE e a Dell, com o projeto Aulas Interativas, e a própria USP, com a Univesp.

Sob a minha perspectiva, o mais importante do que foi exposto até agora é que essas ideias \emph{já estão sendo colocadas em prática}, no CEPA e \emph{com a minha participação}. Ou seja, são propostas com chances reais de se concretizarem. Algumas podem ser alteradas ou adaptadas, ou mesmo tornarem-se obsoletas. Mas a ideia fundamental, {\bfseries desenvolver e difundir conhecimento na criação e produção de Objetos de Aprendizagem} (ou seja lá que nome você prefere), será uma demanda da sociedade no futuro próximo. E abster-me disso é um erro que eu não posso cometer.


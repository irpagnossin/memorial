\begin{thebibliography}{+++}

	\bibitem{pagnossin-2005-1} I. R. Pagnossin, E. C. F. da Silva, A. A. Quivy, S. Martini e C. S. Sergio, \textsl{The quantum mobility of a two-dimensional electron gas in selectively doped GaAs/InGaAs quantum wells with embedded quantum dots}, J. Appl. Phys. \textbf{97}, 113709 (2005).
	
	\bibitem{pagnossin-2004-2} XVII Encontro Nacional de Física da Matéria Condensada, \textsl{A influência de pontos-quânticos de InAs sobre a mobilidade quântica de gases bidimensionais de elétrons confinados em poços-quânticos de GaAs/InGaAs seletivamente dopados}, 2004 (Poços de Caldas).
	
	\bibitem{pagnossin-2005-2} XVIII Encontro Nacional de Física da Matéria, \textsl{The influence of strain fields around InAs quantum-dots on the transport properties of a two-dimensional electron gas confined in GaAs/InGaAs wells}, 2005 (Santos);
	
	\bibitem{pagnossin-2005-3} 12th Brazilian Workshop on Semiconductor Physics, \textsl{The influence of strain fields around InAs quantum-dots on the transport properties of a two-dimensional electron gas confined in GaAs/InGaAs wells}, 2005 (São José dos Campos);
	
	\bibitem{pagnossin-2004-1} I. R. Pagnossin, \textsl{Propriedades de transporte elétrico de gases bidimensionais de elétrons nas proximidades de pontos-quânticos de InAs}, Dissertação de Mestrado, IFUSP, São Paulo (2004).
	
	\bibitem{silva-2003} M. J. da Silva, A. A. Quivy, S. Martini, T. E. Lamas, E. C. F. da Silva e J. R. Leite, \foreign{InAs/GaAs quantum dots optically active at \unit{1.5}{\micro\metre}}, Appl. Phys. Lett. \textbf{82}, 2646 (2003).
	
	\bibitem{johnson-1992} B. L. Johnson, C. Barnes, G. Kirczenow, \textsl{Theory of the Hall effect in two-dimensional quantum-dot arrays}, Phys. Rev. B \textbf{46}, 15302 (1992).
	
	\bibitem{pagnossin-2006} 28th International Conference on the Physics of Semiconductors, \textsl{Quantum Hall effect in bilayer system with array of antidots}, 2006 (Áustria).
	
	\bibitem{pagnossin-2008-1} I. R. Pagnossin, A. K. Meikap, A. A. Quivy, G. M. Gusev, \textsl{Electron dephasing scattering rate in two-dimensional GaAs/InGaAs heterostructures with embedded InAs quantum dots}, J. Appl. Phys. \textbf{104}, 073723 (2008).
	
	\bibitem{pagnossin-2008-2} I. R. Pagnossin, A. K. Meikap, T. E. Lamas, G. M. Gusev, J. C. Portal, \textsl{Anomalous dephasing scattering rate of two-dimensional electrons in double quantum well structures}, Phys. Rev. B, Condensed Matter and Materials Physics \textbf{78}, 115311 (2008). 
	
	\bibitem{pagnossin-2007} 13th Brazilian Workshop on Semiconductor Physics, \textsl{Weak localization and interaction effects in GaAs/InGaAs heteroestructures with nearby quantum-dots}, 2007 (São Paulo).
	
	\bibitem{tidia} TIDIA-Ae (Tecnologia da Informação no Desenvolvimento da Internet Avançada --- Aprendizado Eletrônico), em \url{http://tidia-ae.usp.br/portal}.
	
	\bibitem{applets} Alguns \foreign{applets} desenvolvidos na fase inicial do meu trabalho no CEPA: \url{http://goo.gl/yLLcg}, \url{http://goo.gl/LrdHn} e \url{http://goo.gl/s2igw}.
	
	\bibitem{irpagnossin-stoa} Relatórios sobre a utilização de Java e SVG como plataforma para o desenvolvimento de Objetos de Aprendizagem, em \url{stoa.usp.br/irpagnossin/files}, item ``Apresentação e relatórios técnicos''.
	
	\bibitem{curso-LaTeX} Curso \textsl{Usando \LaTeX; pensando \TeX}, em \url{http://goo.gl/bz6ST} (2011.06.15).
	
	\bibitem{mit} Cursos a distância do MIT (Massachusetts Institute of Technology), em \url{ocw.mit.edu} (2011.06.15).
	
	\bibitem{redefor} Rede São Paulo de Formação Docente, em \url{redefor.usp.br} (2011.06.15).
	
	\bibitem{aulas-interativas} Por exemplo, ``Lousas digitais ligadas à internet são usadas em escolas do interior de São Paulo'', em \url{http://goo.gl/T3dz0} (2011.06.15).
	
	\bibitem{ipad} Veja, por exemplo, \textsl{Escola brasileira substitui apostilas de papel por iPad}, em \url{http://goo.gl/eJ1pP}, \textsl{iPad: escola do Ceará vai utilizar o tablet como ferramenta de ensino}, em \url{http://goo.gl/2BlMK}, ou \textsl{New York orders thousands of iPads for schools}, em \url{http://goo.gl/Pkecg} (2011.06.15).
	
	\bibitem{melare-2009} Guia didático sobre as tecnologias da comunicação e informação, Daniela Melaré Vieira Barros, Vieira e Lent, Rio de Janeiro, 2009.
	
	\bibitem{homem-virtual} Projeto Homem Virtual, em \url{projetohomemvirtual.com.br} (2011.06.15).
	
	\bibitem{geogebra} Geogebra, \foreign{software} de geometria dinâmica, em \url{http://www.geogebra.org/cms/} (2011.06.15).
	
	\bibitem{igeom} iGeom, \foreign{software} de geometria dinâmica desenvolvido pelo Prof. Dr. Leônidas de Oliveira Brandão, do Instituto de Matemática e Estatística da USP, em \url{www.ime.usp.br/~leo/imatica/igeom} (2011.06.15).
		
	\bibitem{coelho-sabido} Coelho Sabido, \foreign{software} educativo para o ensino fundamental, em \url{coelhosabido.com.br} (2011.06.15).
	
	\bibitem{phet} Projeto PhET, em \url{phet.colorado.edu} (2011.06.15).
	
	\bibitem{lock-in} Amplificador \foreign{lock-in} virtual, da National Instruments, em \url{http://goo.gl/HBvKX} (2011.06.15).
	
	\bibitem{falstad} \foreign{Applets} Java para a visualização de conceitos de Física, Matemática e Engenharia, em \url{falstad.com} (2011.06.15).
	
	\bibitem{fendt} \foreign{Applets} Java de Física, traduzido pelo CEPA: \url{www.walter-fendt.de/ph14br} (2011.06.15).	
	
	\bibitem{merlot} Repositório de Objetos de Aprendizagem MERLOT, em \url{www.merlot.org/merlot} (2011.06.15).
	
	\bibitem{phet-research} Lista de artigos sobre as pesquisas acerca das simulações ciradas no projeto PhET\cite{phet}, em \url{phet.colorado.edu/en/research} (2011.06.15).
	
\end{thebibliography}
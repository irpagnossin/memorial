\chapter{Apresenta��o}

Caro Professor, n�o vejo maneira mais honesta de me apresentar do que permitir que algu�m o fa�a por mim. Por isso, gostaria de come�ar este memorial com um depoimento dado pelo Prof. Dr. Ewout ter Haar, com quem tenho o prazer de trabalhar desde dezembro de 2007, em meu perfil no LinkedIn (\url{linkedin.com}). Em suas palavras: (sic)

\begin{quotation}

{\itshape Ivan trabalhou por aproximadamente 2 anos no nosso grupo na �rea de tecnologia educacional. O n�vel e quantidade de conhecimento e habilidade agregado ao grupo foi muito alto. Neste per�odo executava os projetos sugeridos com extrema efici�ncia e originalidade. Mas trouxe tamb�m novos projetos e ideias ao grupo. As discuss�es sobre assuntos t�cnicos foram de alta intensidade e proveitoso para todos. 

O Ivan se tornou especialista em ensino a dist�ncia, anima��es interativas e material did�tico digital de uma forma geral. Entretanto, mais importante do que qualquer habilidade em particular, se mostrou capaz de adquirir novas habilidades e dominar novas tecnologias muito r�pido}

\begin{flushright}
Ewout ter Haar,\\8 de maio de 2009
\end{flushright}

\end{quotation}

Nas p�ginas seguintes, apresentarei minha trajet�ria educacional e profissional, destacando as principais atividades que desenvolvi e mostrando como elas me trouxeram at� aqui. Este documento tamb�m revela, no cap�tulo~\ref{cap:projeto-pesquisa}, o projeto de pesquisa que pretendo desenvolver como docente do IF. E o ap�ndice \ref{cap:resumo} cont�m um resumo das produ��es mais relevantes da minha carreira.

Para registro, este material foi escrito com vistas ao concurso de t�tulos e provas para o provimento de um cargo de Professor Doutor junto ao Departamento de F�sica Experimental do Instituto de F�sica (IF) da Universidade de S�o Paulo (USP), edital IF n�mero 19/2011, de 19 de mar�o de 2011. Ao longo do texto, as palavras destacadas \edital{desta maneira} foram apresentadas no edital como parte da qualifica��o necess�ria para concorrer ao cargo.


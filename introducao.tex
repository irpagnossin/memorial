\chapter{Apresentação}

Caro Professor, não vejo maneira mais honesta de me apresentar do que permitir que alguém o faça por mim. Por isso, gostaria de começar este memorial com um depoimento dado pelo Prof. Dr. Ewout ter Haar, com quem tenho o prazer de trabalhar desde dezembro de 2007, em meu perfil no LinkedIn (\url{linkedin.com}). Em suas palavras: (sic)

\begin{quotation}

{\itshape Ivan trabalhou por aproximadamente 2 anos no nosso grupo na área de tecnologia educacional. O nível e quantidade de conhecimento e habilidade agregado ao grupo foi muito alto. Neste período executava os projetos sugeridos com extrema eficiência e originalidade. Mas trouxe também novos projetos e ideias ao grupo. As discussões sobre assuntos técnicos foram de alta intensidade e proveitoso para todos. 

O Ivan se tornou especialista em ensino a distância, animações interativas e material didático digital de uma forma geral. Entretanto, mais importante do que qualquer habilidade em particular, se mostrou capaz de adquirir novas habilidades e dominar novas tecnologias muito rápido}

\begin{flushright}
Ewout ter Haar\\8 de maio de 2009
\end{flushright}

\end{quotation}

Nas páginas seguintes, apresentarei minha trajetória educacional e profissional, destacando as principais atividades que desenvolvi e mostrando como elas me trouxeram até aqui. Este documento também revela, no capítulo~\ref{cap:projeto-pesquisa}, o projeto de pesquisa que pretendo desenvolver como docente do IF. E o apêndice \ref{cap:resumo} contém um resumo das produções mais relevantes da minha carreira.

Para registro, este material foi escrito com vistas ao concurso de títulos e provas para o provimento de um cargo de Professor Doutor junto ao Departamento de Física Experimental do Instituto de Física (IF) da Universidade de São Paulo (USP), edital IF número 19/2011, de 19 de março de 2011. Ao longo do texto, as palavras destacadas \edital{desta maneira} foram apresentadas no edital como parte da qualificação necessária para concorrer ao cargo.


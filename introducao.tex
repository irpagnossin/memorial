\section{Apresentação}

Caro Professor, não vejo maneira mais honesta de me apresentar do que permitir que alguém o faça por mim. Por isso, gostaria de começar este memorial com um depoimento dado pelo Prof. Dr. Ewout ter Haar, docente do Instituto de Física da Universidade de São Paulo (USP), com quem tenho o prazer de trabalhar. Em suas palavras: (sic)

\begin{quotation}\small

{\itshape Ivan trabalhou por aproximadamente 2 anos no nosso grupo na área de tecnologia educacional. O nível e quantidade de conhecimento e habilidade agregado ao grupo foi muito alto. Neste período executava os projetos sugeridos com extrema eficiência e originalidade. Mas trouxe também novos projetos e ideias ao grupo. As discussões sobre assuntos técnicos foram de alta intensidade e proveitoso para todos. 

O Ivan se tornou especialista em ensino a distância, animações interativas e material didático digital de uma forma geral. Entretanto, mais importante do que qualquer habilidade em particular, se mostrou capaz de adquirir novas habilidades e dominar novas tecnologias muito rápido}

\begin{flushright}
Ewout ter Haar\\8 de maio de 2009
\end{flushright}

\end{quotation}

Este depoimento está disponível na \href{http://br.linkedin.com/in/irpagnossin}{minha página} do LinkedIn, cujo link pode ser acessado na versão digital deste documento, disponível em  \url{http://cepa.if.usp.br/ivan/memorial.pdf}.

Nas páginas seguintes, apresentarei minha trajetória acadêmica e profissional, destacando as principais atividades que desenvolvi e mostrando como elas me trouxeram até aqui. O apêndice \ref{cap:resumo} contém um resumo das produções mais relevantes da minha carreira.

Para registro, este material foi escrito com vistas ao concurso para a contratação de um docente por prazo determinado junto ao Departamento de Física dos Materiais e Mecânica (DFMT) do Instituto de Física (IF) da USP, edital IF-04/2014, de 30 de janeiro de 2014.


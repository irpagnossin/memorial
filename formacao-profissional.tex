\section{Formação profissional}
\label{cap:formacao-profissional}

Minha formação profissional distribuiu-se em três vertentes: eletrônica, desenvolvimento de \foreign{software} e, não menos importante, atendimento ao público. Delas, a segunda foi a que mais influenciou minha formação acadêmica.

\subsection{Liceu de Artes de Ofícios de São Paulo}
\label{sec:liceu}

Minha formação profissional começou com o colégio técnico profissionalizante em eletrônica, no Liceu de Artes e Ofícios de São Paulo, de 1991 a 1995, e paralelamente a ele, auxiliando no empreendimento comercial de meus pais, onde tive meu primeiro contato com o atendimento ao público.

À primeira vista, o tino para com o público pode parecer uma característica dispensável para alguém com uma formação majoritariamente científica e técnica, como a minha, mas aprendi que esta habilidade é inestimável no trabalho em equipe. É, portanto, uma qualidade que prezo.

Durante o colégio técnico, aprendi muito sobre eletrônica, tanto sobre a parte prática quanto sobre a teórica. Mas eu era apenas uma aluno mediano, com dificuldades medianas para apreender os conceitos ensinados. Isto mudou em 1994, quando comecei a estudar cálculo diferencial e integral por conta própria. Esta foi uma de minhas maiores conquistas pessoais, em muito responsável pelas minhas escolhas futuras, dentre elas toda a formação acadêmica descrita na seção anterior.

Nesta época tive meu primeiro contato formal com o desenvolvimento de \foreign{software} (PASCAL), e embora eu já exibisse alguma admiração pela ideia, obtive apenas resultados medianos, a exemplo das demais disciplinas. Curiosamente, desenvolvi, como trabalho da disciplina, um \foreign{software} para explicar as operações de diferenciação e integração (que, entre meus colegas, eu era o único que conhecia): um prenúncio do que eu faria anos mais tarde (seção~\ref{sec:cepa}). Foi nesta época também que desenvolvi práticas de \edital{desenho técnico} e artístico, que emprego ainda hoje.

\subsection{Telemática Sistemas Inteligentes}

No último ano do Ensino Médio (1995), eu fiz um estágio (meu primeiro emprego registrado) na Telemática Sistemas Inteligentes Ltda, também conhecida como Icatel, e responsável pela manutenção de grande parte dos telefones públicos da cidade de São Paulo. Ali coloquei em prática os conhecimentos práticos adquiridos, consertando placas de circuito integrado dos extintos telefones públicos a ficha. Mas não tirei grandes proveitos: na época, minha única preocupação (lamentavelmente) era cumprir as horas do estágio para concluir o ensino técnico.

Quando terminei o estágio e o colégio técnico, fui fazer cursinho (1996). Esta parte não se encaixa bem nem na formação acadêmica nem na profissional, mas foi um período muito importante, pois consolidou os conhecimentos teóricos que eu havia desenvolvido no Ensino Médio, além de corrigir as falhas de formação básica inerentes ao colégio técnico (com muito tempo investido em disciplinas relativas à eletrônica, as disciplinas básicas são prejudicadas). De fato, minha classificação no vestibulinho para o Liceu de Artes e Ofícios, na Escola Técnica Estadudal de São Paulo e no Instituto Tecnológico de Osasco (ITO) foi apenas suficiente para me permitir entrar, e não fui aprovado no vestibulinho para a Escola Técnica \emph{Federal} de São Paulo. Mas quando fiz o vestibular, cinco anos mais tarde, fui aprovado em 14\textordmasculine\ na USP para Bacharelado em Física, em 1\textordmasculine\ na UNESP para Ciências da Computação e em 1\textordmasculine\ na classificação geral do FITO (Faculdade Instituto Tecnológico de Osasco). Também fui aprovado  para a UNICAMP (aparentemente em 51\textordmasculine\ na classificação geral, mas não tenho certeza desta informação). Resumindo, os anos de 1991 a 1995 foram de grande crescimento intelectual e profissional, e o cursinho é parte importante deste processo.

\subsection{Iniciação científica}

Logo que comecei a graduação, a iniciação científica no Laboratório de Física de Plasmas tornou-se minha única ocupação profissional (veja a seção~\ref{sec:graduacao} para mais detalhes). Isto durou até meados de 1998, quando fui aprovado, em 32\textordmasculine, num concurso público para técnico bancário na Caixa Econômica Federal (CEF).

\subsection{Caixa Econômica Federal}
\label{sec:cef}

Na CEF voltei a desenvolver a habilidade de lidar com o público: eu fui inicialmente designado para o setor de FGTS (Fundo de Garantia do Tempo de Serviço), onde ocorriam os mais distintos e complexos problemas. Trabalhei também como caixa, com empréstimos pessoais e estudantís, com financiamentos de habitação e com aplicações, mas foi no FGTS que me destaquei e me especializei, criando procedimentos e mecanismos para otimizar o atendimento daquele setor.

Mas a contribuição mais importante desse período foi a possibilidade de eu comprar meu primeiro computador, o que iniciou a trajetória que percorro até hoje (com salário de bolsista do CNPq isto teria sido impossível). Foi graças a ele que eu aprendi \foreign{hardware} de PC, \LaTeX, Windows e Linux (na época em que se difundia fora do meio acadêmico), AutoCAD, CorelDraw, MathCAD, Mathematica, Matlab, Microcal Origin, Photoshop, 3D Studio e tantos outros. Este equipamento durou até o final do meu mestrado, e lembro-me, com saudade, de tê-lo usado para escrever minha dissertação.

Permaneci na Caixa Econômica Federal até o começo de 2001, quando então fui contratado para desenvolver \foreign{software} na Eletropiezo Indústria e Comércio Ltda.

\subsection{Eletropiezo Indústria e Comércio Ltda}
\label{sec:eletropiezo}

Em abril de 2001, por indicação de um colega da graduação, fui contratado pela Eletropiezo Indústria e Comércio Ltda, uma empresa que produz \foreign{software} para atendimento telefônico (URA, de Unidade de Resposta Audível). Foi neste meio que comecei a programar comercialmente, e passei a ter um tutor na área de programação de computadores: o colega e amigo Gerson de Souza Faria.

Aprendi a programar em T-REXX, uma linguagem proprietária da IBM usada para produzir URA, especificamente para o único projeto de URA IBM em Windows no Brasil, utilizando uma ferramenta chamada DirectTalk (hoje parte do pacote Websphere da IBM). Este trabalho foi desenvolvido para a Fidelity International Systems (FIS), que administra cartões de inúmeros bancos e agentes financeiros, como o Banco Itaú, Panamericano, e até bancos menos conhecidos, como Rural, que ganhou notoriedade em 2006 por abrigar contas usadas em escândalos de corrupção, como o "mensalão".

Aprendi muitas técnicas novas de programação, os princípios da programação orientada a objetos, bem como a trabalhar em equipe e sob a pressão de prazos e responsabilidades: na graduação um erro custava nota; ali custava --- muito --- dinheiro.

Deixei a empresa no início de 2002, para começar o mestrado, mas continuei dando suporte técnico significativo até muito recentemente, pois acabei me tornando um dos poucos profissionais capacitados para este trabalho no Brasil. Para isso precisei abrir uma empresa de desenvolvimento de \foreign{software}, a Cagnotto \& Pagnossin Serviços de Informática Ltda.

O projeto foi um sucesso para todos os envolvidos e manteve-se ativo até dezembro de 2010 (se você, leitor, tem um cartão de crédito, provavelmente já foi atendido por essa URA), quando então foi integralmente substituído por uma tecnologia mais recente.

\subsection{Cagnotto \& Pagnossin Ltda}

Esta é a empresa da qual sou dono. Ela foi inicialmente aberta, em novembro de 2005, para a prestação de serviço de desenvolvimento de software e suporte técnico das URA da FIS. Desde então e até a presente data esta empresa tem prestado serviços para outros clientes, como o Instituto de Pesquisas Eldorado (projeto Aulas Interativas, mais a frente), a Superintendência de Tecnologia da Informação (STI), a Fundação de Apoio à USP (FUSP) e o próprio IFUSP, dentre outras, sempre na área de serviços de informática, especialmente aqueles ligados à Educação.

\subsection{Centro de Ensino e Pesquisa Aplicada}
\label{sec:cepa}

Minhas atividades no Centro de Ensino e Pesquisa Aplicada (CEPA) representam a confluência das minhas trajetórias acadêmica e profissional, e certamente consistem nas minhas mais relevante contribuições para a sociedade e para a USP.

No final de 2007 eu estava descontente com a morosidade da minha pesquisa (doutorado) e com os resultados dela. Não bastasse isso, o prosseguimento padrão seria conseguir uma bolsa de pós-doutoramento, uma ideia que não me agradava.

Nesta ocasião e por intermédio da \profa\ Euzi, conheci o Prof. Dr. Gil da Costa Marques, criador e responsável por um grupo do Departamento de Física Experimental do IFUSP dedicado à criação de materiais didáticos, o CEPA. Ele precisava de um programador para desenvolver \foreign{applets} Java de simulações de fenômenos físicos, como parte do projeto TIDIA-Ae \cite{tidia}, e minha formação acadêmica e profissional fazia de mim a pessoa certa para o trabalho.

Para mim era uma conjunção favorável: congregar Física e desenvolvimento de \foreign{software}, as duas principais áreas nas quais eu vinha investindo há dez anos. Além disso, a confecção da minha dissertação de mestrado e da minha tese de doutorado desenvolveram em mim a capacidade de criar ilustrações, animações e simulações agradáveis aos olhos (\edital{ferramentas de produção gráfica}); e mais importante, a capacidade de simplificar a apresentação de ideias complexas.

E assim comecei a trabalhar no CEPA, com bolsa de capacitação técnica nível TT-4 da FAPESP, em dezembro de 2007. Nos primeiros seis meses eu trabalhei sozinho, pois era o único programador de simulações da equipe, e desenvolvi \foreign{applets} sobre campos vetoriais, integrais de linha, lançamento balístico com resistência do ar, ângulos de Euler (o primeiro que envolvia o uso de programação tridimensional), entre vários outros.% (veja a seção \ref{sec:adendo}).

No início esses \foreign{applets} distinguiam-se dos demais, encontrados na Internet, apenas pelo \edital{\foreign{design} da interação} (ou, de forma mais ampla, a \edital{experiência do usuário}), embora ainda sutilmente. Mas esta preocupação guiou meu trabalho com \foreign{applets} nos meses seguintes, onde procurei desenvolver métodos para trabalhar conjuntamente com artistas, que então ficariam responsáveis pela parte visual. A ideia era que um recurso \emph{didático} precisava não apenas passar o conceito a que se propunha, mas tão importante quanto isso, precisava também cativar o usuário (conceito conhecido como ``onboarding''). Os resultados deste trabalho podem ser obtidos no meu perfil na \edital{comunidade social} Stoa \cite{irpagnossin-stoa}.

\subsubsection{O projeto \textsl{Usando \LaTeX; pensando \TeX}}
\label{sec:latex}

Ainda no primeiro semestre de 2008, o Prof. Gil pediu que eu montasse um curso sobre \LaTeX, um \edital{sistema de produção de documentos} que eu havia aprendido a utilizar na graduação, para os relatórios de laboratório. Desta encomenda surgiu o curso ``Usando \LaTeX; pensando \TeX'', que foi oferecido para a comunidade USP através da então Coordenadoria de Tecnologia da Informação (CTI), hoje Superintendência de Tecnologia da Informação (STI).

O curso foi concebido inicialmente para ser semi-presencial, com 20 horas de duração. O enfoque dele era totalmente prático (usando \LaTeX), mas explorava profundamente os conceitos fundamentais do sistema (pensando \TeX). E por ser semipresencial, todo o conteúdo do curso, como tutoriais, apresentações, atividades práticas, exercícios e lições (muitos deles com avaliação automática e um extenso e cuidadoso sistema de \foreign{feedbacks}), foi disponibilizado no sistema de gerenciamento de cursos Moodle, que aprendi a usar num curso oferecido pela CTI. Não obstante isso, a parte não-presencial valia-se também das mais modernas ferramentas de aprendizado colaborativo e da \edital{web 2.0}, como fóruns, chats e \edital{wikis}. Já a parte presencial do curso ocorria na sala multimeios do IFUSP, que contava com 25 \foreign{notebooks} e uma lousa eletrônica, equipamentos que foram realmente utilizados no curso.

A primeira turma oficial foi aberta no segundo semestre de 2008, tendo eu como professor (um trabalho \foreign{pro bono}), minha colega Juliana Giordano como tutora e meu colega Marcelo Alves como \foreign{designer} instrucional\footnote{A primeira turma de fato (experimental) ocorreu em julho e agosto de 2008, e contou com a presença do Prof. Gil e de funcionários do CEPA, da CTI e do IFUSP que precisavam daqueles conhecimentos, principalmente para auxiliar os professores na escrita de seus artigos científicos.}. E foi um grande sucesso, mostrando que havia de fato interesse por um curso assim na USP: em menos de duas horas após a abertura das inscrições, na comunidade Stoa, as 25 vagas já estavam completas; e antes do final daquele dia, já havia mais de 150 inscritos.

Mas o curso exigia muito dos alunos, pois concentrava-se em atividades: metade de \emph{toda} aula presencial era composta por exercícios. E no Moodle existiam inúmeras atividades e exercícios para serem feitas, com graus de complexidade crescentes. De fato, apenas 11 pessoas concluiram a primeira turma, e cada uma recebeu um diploma endossado pelo Prof. Gil, então coordenador da CTI.

Em seguida o curso foi reformulado e expandido para 24 horas, e uma nova turma foi oferecida no primeiro semestre de 2009 (eu novamente como professor). Delas, 7 chegaram ao final.

Este curso foi uma das minhas mais significativas produções no CEPA e ele continua disponível através da Internet \cite{curso-LaTeX}, mas nenhuma outra turma foi oferecida (ainda há procura), pois a proposta inicial era que se tornasse um curso auto-instrucional a distância. Ademais, um novo projeto entrava em cena, que requereria toda a minha atenção: o projeto Aulas Interativas.

\subsubsection{O projeto Aulas Interativas}
\label{sec:aulas-interativas}

Ainda no primeiro semestre de 2009, o CEPA foi procurado pela \profa\ Maria Alice Carraturi Pereira, então Assessora de Tecnologia Educacional da Secretaria de Estado da Educação (SEE), para um projeto em parceria com a Dell Computadores do Brasil. A proposta era instalar uma lousa eletrônica em cada uma das 26 escolas públicas da cidade de Hortolândia, próxima a Campinas, e caberia ao CEPA produzir os conteúdos interativos para as lousas, para as disciplinas de Língua Portuguesa e Matemática da 6\textordfeminine\ série do Ensino Fundamental e do 1\textordmasculine\ ano do Ensino Médio.

Na verdade, o CEPA fora convidado a apresentar uma proposta de aula interativa, com lousa eletrônica (concorríamos com outras empresas, como Clickideia, Klick educação e Fundação Conesul). E coube a mim montar essa aula e apresentá-la\footnote{Cabe mencionar que as habilidade desenvolvidas na criação do curso de \LaTeX\ contribuiram bastante, especialmente no que se referia a falar em público, uma tarefa que eu passei a não ter dificuldade.} para membros da SEE e da Dell. O tópico, escolhido pela SEE, era ``as relações métricas do triângulo-retângulo''. Ao montar a aula, tomei o cuidado de partir de conceitos cotidianos (algo que em pedagogia se chama construcionismo), e usando a lousa eletrônica, mostrei como a dedução das relações métricas do triângulo-retângulo tornavam-se simples quando valendo-se de um \foreign{software} que eu produzi especialmente para esta apresentação. A proposta agradou, pois o CEPA foi escolhido para o trabalho (e eu fui convidado a reapresentar esta aula inúmeras outras vezes).

Curiosamente, convém mencionar que este projeto quase não foi concretizado devido às falhas encontradas nos cadernos de Geografia, naquele ano, e que causaram a queda da então Secretária da Educação.

Assim o projeto começou, pouco antes do segundo semestre de 2009, com enfoque voltado para a produção de \foreign{softwares} interativos para lousas eletrônicas. E eu passei a liderar uma pequena equipe de programadores (com estagiários da USP, inclusive), bem como a coordenar a produção desse material com a equipe de arte do CEPA, até então ausentes na criação desse tipo de conteúdo (lembre-se: até então eu era o único programador do CEPA e trabalhava sozinho). Mais especificamente, eu me tornei responsável pela produção de todo o material de Matemática, e embora oficialmente eu não fora indicado para propor os conteúdos interativos, com base nos cadernos desenvolvidos pela SEE, pois minha formação não era a de matemático, eu muito influenciei esse material, num trabalho conjunto com os professores indicados pela SEE (e que haviam criado os cadernos) e com os Professores Coordenadores das Oficinas Pedagógicas (PCOP).

Atuei também na capacitação dos PCOP quanto ao uso dos softwares produzidos e cheguei a treinar professores da rede pública, usuários das lousas e do material por nós criado, em uma aula temática criada para a abertura oficial do projeto, na SEE, em novembro de 2009. Esta apresentação teve as presenças ilustres do então Secretário de Educação, Paulo Renato Souza, e do fundador da Dell, Michael Dell.

Através deste projeto eu aprendi novas tecnologias como Flash, ActionScript e lousa eletrônica; aprendi a trabalhar com pessoas de outras áreas, como os professores da SEE e os ilustradores do CEPA; desenvolvi em mim a capacidade de liderar uma equipe, e coordená-la para atingir um objetivo (dentro do prazo, sempre que possível); e por que não dizer: tive a oportunidade de conhecer pessoalmente as precárias situações em que os professores de Hortolândia bravamente exercem sua profissão, e apesar disso, ainda assim observar o brilho nos olhos de cada aluno(a) com as novidades que levávamos para eles.

Para o CEPA, foi a oportunidade de evoluir --- muito --- em termos de profissionalismo, e prepará-lo para o que estava por vir: os projetos Univesp e Redefor. A participação do CEPA, na produção de material, terminou em dezembro de 2010, mas o projeto continuou em andamento por mais algum tempo. Em 2011 a Unesco avaliou o projeto, tendo evidenciado bons resultados, que foram então publicados na edição de junho da Revista Época daquele ano. No ano seguinte, o Governo do Estado de São Paulo abriu uma licitação para a produção em larga escala de mais recursos para lousas eletrônicas, desta vez abrangendo todo o currículo oficial.

Em tempo, é importante observar que a escolha pela tecnologia Flash, feita por mim (eu era o único programador do CEPA), foi crucial para que o projeto fosse cumprido. Que a verdade seja dita: de um lado, o governo de José Serra impusera que o projeto deveria estar pronto até o final de 2010 (época de eleição); do outro, a SEE exigira um ano letivo completo de material produzido. Esses eram os termos para a Dell. Para o CEPA significava um prazo irrisório. Tivesse eu escolhido utilizar Java, nós não teríamos chegado ao final do projeto, pois a produtividade dele é bem menor que a do Flash.

\subsubsection{Os projetos Univesp e Redefor}

Após a criação do convênio USP/Univesp (Universidade Virtual do Estado de São Paulo), em março de 2010, com a outorga do então Governador José Serra\footnote{A aprovação deste projeto vinha sendo adiada há pelo menos um ano por pressões de setores da sociedade contrários à educação a distância.}, a USP ficou responsável pela criação do curso semi-presencial de \href{http://licenciaturaciencias.usp.br/}{Licenciatura em Ciências}. A primeira turma teve início no segundo semestre de 2010 e atualmente há quatro turmas em curso.

O CEPA ficou responsável pelo desenvolvimento dos recursos educacionais do curso, dentre os quais os softwares educacionais que minha equipe produzia. Isso era (e ainda é) feito em parceria com docentes da USP e professores de atividade. Todos esses materiais foram disponibilizados aos alunos por meio de um ambiente virtual de aprendizagem (AVA) Moodle.

Oficialmente, minha atribuição nesse projeto é a de \emph{professor de atividades}: eu proponho atividades de aprendizagem, na forma de softwares educacionais interativos, para as disciplinas Fundamentos de Matemática I e II, Dinâmica do Movimento dos Corpos e Eletromagnetismo. Não obstante isso, eu mantenho minha atribuição anterior, de \emph{coordenador}: sou responsável por executar as demandas de softwares educacionais requisitados pelos docentes e outros professores de atividades, geralmente por intermédio dos designers instrucionais do CEPA.

Como resultado desse trabalho, até o momento nós temos aproximadamente 180 recursos educacionais interativos produzidos, de várias áreas (Biologia, Física, Matemática, Química e até Educação).

Mas eu executo uma outra tarefa, a de \emph{pesquisador}: devido à autonomia e à proximidade com a sala de aula que o projeto me proporcionou, eu pude começar a fazer pesquisa sobre o uso dos recursos educacionais que criamos, visando aprimorar a aprendizagem com eles, ainda que este não fosse o objetivo do CEPA neste projeto.

Inicialmente, esse trabalho foi desenvolvido individualmente, mas com o tempo foi possível angariar colaboradores. Atualmente, tenho trabalhos desenvolvidos com quatro designers instrucionais, três professoras das disciplinas de Matemática, uma coordenadora de pólo do USP/Univesp e um aluno de mestrado.

Os resultados mais notórios dessas pesquisas foram apresentados em vários eventos, como CIAED (2011 e 2012), APPLETS (2011), SIED:EnPED (2012), TicEduca (2012), WebCurrículo (2012) e GPIMEM (2013), sendo que alguns deles foram publicados nos anais dos eventos (veja a seção \ref{cap:resumo}). Agora, esses resultados estão sendo preparados para publicação regular, e com isso espero conseguir de 4 a 6 publicações este ano. Essa é uma meta importante, pois efetivaria minha atuação no CEPA como \emph{pesquisa \textbf{e} desenvolvimento} (P\&D).

Além da P\&D, grande parte do meu esforço atual é dedicado à otimização dos processos internos do CEPA. Um exemplo disso é a implantação do sistema de desenvolvimento ágil de projetos chamado Scrum, que eu trouxe para o CEPA no final de 2010. Outro é a utilização do padrão SCORM (\foreign{Sharable Content Object Reference Model}) nos softwares que desenvolvemos, também introduzido por mim em meados de 2010, e que eu já vinha estudando um ano antes. No CEPA ele é empregado parcialmente e já rendeu uma bolsa de treinamento técnico para um dos integrantes da minha equipe (oficialmente, ele é um aluno do Prof. Dr. Ewout ter Haar).

O SCORM é um padrão de e-learning já antigo, mas a sua importância nesse projeto se deve ao fato de que, com ele, podemos criar cenários de avaliação mais ricos que aqueles das listas de exercícios que os AVA geralmente permitem.

O CEPA e, particularmente, a equipe de criação de objetos de aprendizagem, produziram material para outro grande projeto, também em parceria com a Secretaria de Estado da Educação: o \href{http://redefor.usp.br/}{Redefor}, ou Rede São Paulo de Formação Docente, que consiste numa série de cursos de formação continuada para professores da rede pública de ensino do Estado de São Paulo. Neste projeto eu atuei apenas coordenador da minha equipe, procurando garantir a entrega dos recursos encomendados.

\subsubsection{Resumo}

Minha atuação no CEPA é a mais diversa possível: não apenas consegui conjugar Física/Matemática e programação, duas áreas que gosto muito, como tenho trabalhado ativamente para estabelecer o CEPA como um modelo de equipe multidisciplinar educacional em P\&D. Pessoalmente, evoluí em todos os aspectos possíveis: como professor, criei cursos e lecionei; como pesquisador, consegui realizar estudos mesmo num cenário não favorável; como coordenador, aprendi a trabalhar em equipe e a coordenar uma equipe; e como desenvolvedor, consegui aplicar as tecnologias digitais na Educação.

\subsection{Senac}

Devido à minha experiência, desenvolvida no CEPA, em agosto de 2012 e em março de 2013, por indicação da \profa\ Vani Kenski, atual coordenadora dos designers instrucionais do CEPA, atuei como docente na disciplina ``Processos de produção de objetos de aprendizagem'' do curso de pós-graduação a distância em \foreign{design} instrucional do Senac/SP (Serviço Nacional de Aprendizagem Comercial). Mais tarde, em meados de 2013, fui contratado pelo Senac para recriar essa disciplina. Todos esses trabalhos já foram concluídos.

\subsection{USP}

Em 20 de fevereiro de 2014, fui aprovado no concurso 04/2014 do IFUSP, para docente por prazo determinado junto ao DFMT. Desde então, tenho atuado também como docente na USP. No primeiro semestre de 2014, fiquei responsável por duas turmas (55 alunos) da disciplina ``Física para Engenharia I'', da Escola Politécnica (carga horária de 36 horas). No segundo semestre de 2014, fiquei responsável por uma turma (47 alunos) da disciplina ``Física para Engenharia II'', da Escola Politécnica (carga horária de 60 horas).

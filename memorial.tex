%-------------------------------------------------------------------------------
% Arquivo: 		memorial.tex
% Autor:			Ivan Ramos Pagnossin
% Descrição:	Memorial escrito para o concurso de admissão de docente no IFUSP
%-------------------------------------------------------------------------------
\documentclass[a4paper,12pt,twoside]{scrartcl}					% Sub-classe koma-script

	%
	% Pacotes necessários.
	%
	\usepackage[brazil]{babel}								% Idioma português
	\usepackage[utf8]{inputenc}							% Caracteres acentuados
	\usepackage[T1]{fontenc}								% Fontes de 16 bits
	\usepackage{fouriernc}
	\usepackage{xcolor}										% Cores
	\usepackage[a4paper, top=3cm, left=3cm, right=2cm, bottom=2cm, includefoot]{geometry}						% Formatação de página
	\usepackage{indentfirst}								% Indentação do primeiro parágrafo de cada seção
	\usepackage{setspace}									% Espaçamento entre-linhas
	\usepackage[squaren,binary]{SIunits}				% Sistema internacional de unidades
	\usepackage{amsmath}										% Environments e comandos da American Mathematical Society	
	\usepackage{paralist}									% Environment "compactitem"
  \usepackage{url}
  \usepackage{fancyhdr}
  \usepackage[symbol,norule,flushmargin]{footmisc}
    \usepackage[colorlinks=true,linkcolor=blue,urlcolor=blue]{hyperref}
      \usepackage{graphicx}

	%
	% Definições úteis.
	%
	\newcommand\foreign[1]{\textnormal{#1}}
	\newcommand\edital[1]{\textnormal{#1}}%\textcolor{green!30!black}{#1}}
	\newcommand\profa{Prof\rlap.\textordfeminine}
	\newcommand\etc{\foreign{etc}}
	\newcommand\dra{Dr\rlap.\textordfeminine}
	\newcommand\ie{\foreign{ie}}

	% Insere minhas definições de formato (REORGANIZAR).
	\input{myPortuges.ldf}

	% Impede que o texto seja uniformemente distribuído na vertical.
	\raggedbottom

	% Espaço entre-linhas: 30% maior que o normal.
	\setstretch{1.3}

	% Configuração manual da divisão silábica de algumas palavras.
	\hyphenation{}

	% Configuração dos cabeçalhos e rodapés
	\pagestyle{fancy}
	\fancyhead{}
	\fancyhead[LE]{I. R. Pagnossin}
	\fancyhead[RE]{}
	\fancyhead[LO]{}
	\fancyhead[RO]{{\footnotesize Memorial circunstanciado}}
	\fancyfoot{}
	\fancyfoot[LE,RO]{\thepage}
	\fancyfoot[RE,LO]{}
	\renewcommand\footrulewidth{0.4pt}
	
	% Configuração das notas de rodapé
	\setfnsymbol{lamport}

	% Configuração da capa
	\title{Memorial circunstanciado}
	\author{Ivan Ramos Pagnossin}
	
	\hyphenation{E-le-trô-ni-ca E-le-tro-pi-e-zo dou-to-ra-do pe-da-gó-gi-co do-mi-nar e-xa-me gra-du-a-ção he-te-ro-es-tru-tu-ras re-gis-tra-do a-pre-sen-ta-ção}
	
%-------------------------------------------------------------------------------
%-------------------------------------------------------------------------------
%	Aqui começa o documento propriamente dito.
%-------------------------------------------------------------------------------
%-------------------------------------------------------------------------------


\begin{document}

	\maketitle
	%\tableofcontents
	%\pagenumbering{roman}

	% Prefácio
	%\pagebreak
	%\pagenumbering{arabic}
	%\setcounter{page}{1}
	
	\chapter{Introdu��o}

O intuito deste memorial � apresentar a minha trajet�ria profissional e educacional, destacando as principais atividades que desenvolvi e enfatizando as influ�ncias delas em minhas atividades atuais e perspectivas para o futuro, no meio acad�mico e fora dele.

Este material foi escrito com vistas ao concurso de t�tulos e provas para o provimento de um cargo de Professor Doutor junto ao Departamento de F�sica Experimental do Instituto de F�sica (IF) da Universidade de S�o Paulo (USP), edital IF n�mero 19/2011, de 19 de mar�o de 2011. Ao longo do texto, as palavras destacadas \edital{desta maneira} foram apresentadas no edital como parte da qualifica��o necess�ria para concorrer ao cargo.

Este documento tamb�m contempla, no cap�tulo~\ref{cap:projeto-pesquisa}, o projeto de pesquisa que pretendo desenvolver como docente do IF.
	\section{Formação acadêmica}

Minha formação acadêmica sempre foi voltada para o estudo da Física Básica e Aplicada, particularmente na área de materiais semicondutores e de fenômenos de transporte eletrônico. No entanto, ela sofreu grande influência da minha formação profissional (seção~\ref{cap:formacao-profissional}) --- bem como a influenciou ---, especialmente no que concerne o desenvolvimento de \foreign{software}.

\subsection{Graduação e iniciação científica}
\label{sec:graduacao}

Minha formação acadêmica começou em fevereiro de 1997, quando fui aprovado, em 14\textordmasculine, no exame vestibular da USP para o curso de Bacharelado em Física, no \foreign{campus} da capital. A escolha pela Física foi moldada, anos antes, pelo contato com circuitos eletrônicos, no Ensino Médio profissionalizante (seção~\ref{sec:liceu}), bem como por uma paixão ainda mais antiga: a aviação (hoje sou piloto de planador). Já a opção pelo bacharelado foi feita pelo desejo de aprender \emph{profundamente} fenômenos físicos e técnicas matemáticas, em muito alimentado pelo aprendizado autodidata de cálculo diferencial e integral, ainda no Ensino Médio.

Na verdade, o fato de eu já conhecer o cálculo diferencial e integral ao começar a graduação ajudou-me enormemente, permitindo-me aproveitar melhor os conceitos e técnicas ensinados, bem como ir além deles, em todas as disciplinas cursadas. De fato, em oposição ao que ocorreu durante o Ensino Médio (veja a seção \ref{sec:liceu}), eu me destaquei em praticamente todas as disciplinas, chegando a atingir médias como 9,5 em disciplinas como Cálculo, Álgebra Linear, Física Atômica e Molecular, entre outras.

O Ensino Médio profissionalizante teve outra clara e positiva influência na minha graduação: logo na primeira semana consegui uma bolsa de iniciação científica do CNPq no Laboratório de Física de Plasmas, orientado pelo Prof. Dr. Ivan Cunha Nascimento, e em muito auxiliado pelo funcionário Juan Iraburu Elizondo. A proposta era a de estudar a chamada curva de \foreign{breakdown}, que caracterizava a formação de plasma no Tokamak. Os resultados deste trabalho foram apresentados no Simpósio de Iniciação Científica da USP de 1998. Permaneci na iniciação científica até meados de 1998, quando então fui contratado pela Caixa Econômica Federal (CEF).

Mais influentes que a iniciação científica foram as disciplinas de introdução à computação e de cálculo numérico, que contribuiram para um novo rumo na minha formação profissional, dali alguns anos (seção \ref{sec:eletropiezo}), e que sigo até hoje. Igualmente importante foi a disciplina de introdução à Física do Estado Sólido, na qual conheci a \profa\ Dr\rlap{\textordfeminine}. Euzi Conceição Fernandes da Silva, que me convidou para fazer a pós-graduação no Departamento de Física dos Materiais (DFMT) do IF da USP (IFUSP) e que muito me auxiliou desde então.

Finalmente, outra grande conquista pessoal ocorrida durante a graduação foi a auto-instrução da língua inglesa. Embora eu tenha feito alguns cursos esporadicamente, foi na graduação que adquiri maturidade intelectual para assimilar a língua, principalmente através dos livros e filmes da videoteca do IFUSP.

Assim, concluí a graduação em 2001 com aproveitamento médio de 83\% e com habilitação em Física Básica. Meu intuito era obter também a habilitação em microeletrônica, mas isto estenderia a graduação por pelo menos mais um ano, o que eu não estava disposto a aceitar, pois já havia gasto um ano extra no Ensino Médio profissionalizante, outro no cursinho e mais outro na graduação (quando migrei do período matutino para o noturno, na ocasião de minha contratação pela CEF). Ademais, a oferta de pós-graduação com a \profa\ Euzi já me levava para a área do transporte eletrônico. Deste modo, desisti da habilitação em microeletrônica.
 
\subsection{Mestrado}

Na época em que me candidatei ao mestrado \foreign{stricto sensu}, em 2001, logo após concluir a graduação, a FAPESP já iniciava seu movimento em prol do doutoramento direto; sem mestrado. No entanto, apesar do meu desespero em avançar na carreira acadêmica e por orientação da \profa\ Euzi, decidi pelo caminho mais longo: o mestrado, na certeza de que era um passo importante que não deveria ser pulado (ainda hoje acredito que esta foi uma escolha acertada).

Minha opção por uma pós-graduação experimental é outra que merece explicação: ao longo da graduação eu percebi que tinha muita facilidade com a teoria, mas nem tanto com a prática. Assim, minha expectativa era que um mestrado experimental me permitiria corrigir este desequilíbrio. Isto realmente aconteceu, mas a minha ``veia teórica'' sempre deu suas contribuições, no mestrado e no doutorado, e ainda hoje é mais expressiva.

A proposta para o mestrado era caracterizar a evolução de pontos-quânticos auto-organizados através de medidas ópticas (fotoluminescência) e de transporte eletrônico (efeitos Hall clássico e quântico e Shubnikov-de Haas) em baixas temperaturas ($\sim\unit{1,4}{\kelvin}$). E deste modo aprendi a manusear nitrogênio e hélio-4 líquidos, bem como equipamentos complexos como criostatos, bombas de vácuo, amplificadores \foreign{lock-in}, espectrômetros, \foreign{lasers} de alta potência \etc. Aprendi também técnicas como litografia, microscopias de varredura (principalmente de força atômica), crescimento epitaxial molecular e confecção de contatos eletrônicos por difusão. Em suma, o mestrado foi um período de intenso aprendizado, como deveria ser.

Como resultado deste trabalho, chegamos à conclusão de que a tensão mecânica acumulada nos pontos-quânticos, por consequência do crescimento epitaxial, afeta as mobilidades dos elétrons. Este foi um resultado inédito na literatura científica (até onde sabemos), o que nos rendeu um artigo \cite{pagnossin-2005-1}, uma exposição dele (pôster) no XVII Encontro Nacional de Física da Matéria Condensada (ENFMC), em 2004 \cite{pagnossin-2004-2}, e, mais tarde, uma versão expandida dele no XVIII ENFMC e no \foreign{12th Brazilian Workshop on Semiconductor Physics} (BWSP), em 2005 \cite{pagnossin-2005-2, pagnossin-2005-3}. Atualmente este trabalho tem 9 citações, segundo o Web of Science. Além disso, este era um resultado importante para o rumo que nosso grupo de pesquisa buscava naquela época: o estudo e confecção de \foreign{lasers} e detectores de infra-vermelho baseados em pontos-quânticos.

Além desses resultados, dois outros destacaram-se: o primeiro foi a dedução matemática da técnica utilizada na análise das oscilações de magnetoresistência (o chamado efeito Shubnikov-de Haas), que aparentemente perdeu-se na literatura (nós nunca a encontramos). Esta dedução está registrada nos apêndices da minha dissertação de mestrado \cite{pagnossin-2004-1}. O segundo foi o desenvolvimento de um script\footnote{Escrito em LabTalk, linguagem de script do \foreign{software} de análise de dados Microcal Origin.} para automatizar parte dessa análise, o que permitiu reduzir o tempo dela em aproximadamente 90\%. Este script, mais a compreensão do método adquirida na dedução matemática dele, permitiu-me desenvolver uma pesquisa informal paralela para estabelecer os limites da técnica e seus efeitos sobre os dados.

Assim, concluí o mestrado em maio de 2004 com resultados empolgantes (algo incomum de acontecer, segundo a \profa\ Euzi) e os apresentei para a banca examinadora, composta pela \profa\ \dra\ Lucy Vitoria Credidio Assali (IFUSP), pelo Prof. Dr. Marcelo Nelson Paez Carreño (da Escola Politécnica da USP) e, claro, pela \profa\ \dra\ Euzi Conceição Fernandes da Silva.

\subsection{Doutorado}

Continuando o trabalho do mestrado, a proposta de pesquisa para o doutorado era caracterizar as possíveis heteroestruturas-base de detectores de infra-vermelho baseados em pontos-quânticos. A discussão, na literatura científica, sobre qual seria a melhor estrutura para este dispositivo, estava no auge. Além disso, nosso grupo de pesquisa havia conseguido um resultado até então dado como impossível: a absorção, por pontos-quânticos, de ondas de infra-vermelho com comprimento de onda de \unit{1,5}{\micro\metre} \cite{silva-2003}. A importância deste resultado, e de todos os estudos que se seguiram, residia no fato de que a fibra óptica utilizada em telecomunicações apresenta um mínimo absoluto de absorção nesta frequência, de modo que dispositivos operando nesta faixa trariam grandes benefícios econômicos.

O estudo começou, então, por experimentar algumas possíveis configurações de heteroestruturas, conforme propostas existentes na literatura científica. A ideia era utilizar nossa já conhecida caracterização eletrônica para determinar as mobilidades dos elétrons e, com isso, identificar a melhor configuração para o dispositivo.

No entanto, aproximadamente um ano após o início do doutorado, a \profa\ Euzi foi para o \foreign{Center for Quantum Devices}, nos EUA, a convite da \profa\ Manijeh Razeghi, e desta maneira fui obrigado a mudar de orientador.

O Prof. Dr. Guennadii Michailovich Gusev, que assumiu a chefia do DFMT com a morte do Prof. Dr. José Roberto Leite, em 2004, cordialmente aceitou orientar-me a partir daí. No entanto, sua linha de pesquisa concentrava-se em fenômenos de Física Básica, como o efeito Hall quântico fracionário, transporte eletrônico em sistemas mesoscópicos, efeitos de \foreign{spin} em sistemas bidimensionais \etc. E deste modo minha pesquisa foi alterada para o estudo de redes de anti-pontos-quânticos.

Esta mudança foi muito benéfica, pois a visão do Prof. Gusev sobre os assuntos da pesquisa era deveras diferente daquele da \profa\ Euzi, de modo que isto me deu perspectivas novas. Ademais, aprendi inúmeras outras técnicas experimentais, como manusear hélio-3, nanolitografia por microscopia eletrônica, confecção de \foreign{gates} de ouro por evaporação, além de formalismos matemáticos como o de Landauer-Büttiker, entre outros. No entanto, a troca de orientador teve um efeito severo sobre minha pesquisa: eu praticamente a desenvolvi sozinho. Embora o Prof. Gusev sempre se dispusesse a discutir qualquer assunto, sua presença na minha pesquisa não era tão frequente quanto a da \profa\ Euzi. Isto prejudicou um pouco a qualidade do trabalho que desenvolvi, mas me tornou mais independente, o que viria a ser fundamental na minha carreira no CEPA (veja a seção~\ref{sec:cepa}).

Durante o desenvolvimento desse trabalho, encontramos, por acaso, evidências experimentais dos chamados estados de borda contra-rotativos, previstos teoricamente em 1992 \cite{johnson-1992}, mas até então não observados. E a partir daí minha pesquisa voltou-se para este assunto.

Entretanto, a construção das amostras requeridas para este estudo estava no limiar da capacidade técnica que tínhamos à disposição, o microscópio eletrônico do Laboratório de Sistemas Integráveis (LSI) da Escola Politécnica. Não obstante isso, nosso acesso a este equipamento era raro, o que tornava deveras demorado obter um conjunto de amostras. Adicione a isto a constante dificuldade em conseguir hélio-4 para os criostatos (a demanda do grupo era grande, em parte devido ao Detector de Ondas Gravitacionais Mário Schenberg, que se preparava para entrar em operação), e o resultado é que nunca conseguimos reproduzir aqueles resultados.

Apesar disso, conseguimos apresentar as primeiras evidências no \foreign{28th International Conference on the Physics of Semiconductors}, o mais importante congresso de Física de Semicondutores, em 2006, na Áustria \cite{pagnossin-2006}.

Passados dois anos de doutorado eu estava com um grande problema nas mãos: minha pesquisa inicial, sobre fotodetectores, havia sido interrompida prematuramente, e aquela sobre os estados de borda contra-rotativos não avançava o suficiente para apresentar uma tese de doutorado. Foi então que procurei o Prof. Dr. Ajit Kumar Meikap, do \foreign{National Institute of Technology}, na Índia (que estava passando uma temporada no Brasil, a convite do Prof. Gusev), e propus que fizéssemos estudos de localização-fraca em amostras mais simples (poços-quânticos duplos e parabólicos), dentre elas aquelas utilizadas no meu mestrado (pontos-quânticos).

O Prof. Meikap havia desenvolvido, durante sua estada no Brasil, todo o ferramental para analisar dados conforme os mais recentes estudos sobre localização-fraca, mas não tinha o que analisar. Eu, por outro lado, tinha um enorme conjunto de medidas já prontas, e inúmeras outras que podiam ser feitas com facilidade, pois na época eu estava fazendo um estágio de dois meses e meio no \foreign{Grenoble High Magnetic Field Laboratory}, na França, sob supervisão do Prof. Dr. Jean-Claude Portal, com equipamentos à minha disposição quase exclusiva.

Esta parceria rendeu dois artigos \cite{pagnossin-2008-1, pagnossin-2008-2} e uma exposição no \foreign{13th} BWSP, em 2007 \cite{pagnossin-2007}. Atualmente, um desses trabalhos tem 11 citações, segundo o Web of Science.

Na tese de doutorado apresentei, então, três conjuntos de resultados: aqueles dos fotodetectores (embora incompletos, já era possível tirar algumas conclusões que guiassem a confecção de fotodetectores baseados em pontos-quânticos), aqueles dos estados de borda contra-rotativos (os que eu mais gostei, apesar de tudo) e aqueles relacionados às medidas de localização-fraca.

A defesa da tese de doutoramento ocorreu em abril de 2008, tendo como banca examinadora o Prof. Dr. Antônio Carlos Seabra (EPUSP), o Prof. Dr. Eliermes Arraes Meneses (UNICAMP), a \profa\ \dra\ Euzi Conceição Fernandes da Silva (IFUSP), o Prof. Dr. Fernando Iikawa (UNICAMP) e a \profa\ \dra\ Lucy Vitória Credidio Assali (IFUSP).

	\chapter{Forma��o profissional}
\label{cap:formacao-profissional}

Minha forma��o profissional distribuiu-se em tr�s vertentes: eletr�nica, desenvolvimento de \foreign{software} e, n�o menos importante, atendimento ao p�blico. Delas, a segunda foi a que mais influenciou minha forma��o acad�mica.

\section{Liceu de Artes de Of�cios de S�o Paulo}
\label{sec:liceu}

Minha forma��o profissional come�ou com o col�gio t�cnico profissionalizante em eletr�nica, no Liceu de Artes e Of�cios de S�o Paulo, de 1991 a 1995, e paralelamente a ele, auxiliando no empreendimento comercial de meus pais, onde tive meu primeiro contato com o atendimento ao p�blico.

� primeira vista, o tino para com o p�blico pode parecer uma caracter�stica dispens�vel, especialmente para algu�m com uma forma��o majoritariamente t�cnica e cient�fica, como a minha, mas aprendi que isto n�o � verdadeiro, e confirmo essa certeza todos os dias. �, portanto, uma qualidade que prezo.

Durante o col�gio t�cnico, aprendi muito sobre eletr�nica, tanto sobre a parte pr�tica quanto sobre a te�rica. Mas eu era apenas uma aluno mediano, com dificuldades medianas para apreender os conceitos ensinados. Isto mudou em 1994, quando comecei a estudar c�lculo diferencial e integral por conta pr�pria. Esta foi uma de minhas maiores conquistas pessoais, em muito respons�vel pelas minhas escolhas futuras, dentre elas toda a forma��o acad�mica descrita no cap�tulo anterior.

Nesta �poca tive meu primeiro contato formal com o desenvolvimento de \foreign{software} (PASCAL), e embora eu j� exibisse alguma admira��o pela ideia, obtive apenas resultados medianos, a exemplo das demais disciplinas. Curiosamente, desenvolvi, como trabalho da disciplina, um \foreign{software} para explicar as opera��es de diferencia��o e integra��o (que, entre meus colegas, eu era o �nico que conhecia): um pren�ncio do que eu faria anos mais tarde (se��o~\ref{sec:cepa}). Foi nesta �poca tamb�m que desenvolvi pr�ticas de \edital{desenho t�cnico} e art�stico, que emprego ainda hoje.

\section{Telem�tica Sistemas Inteligentes}

No �ltimo ano do ensino m�dio (1995), eu fiz um est�gio (meu primeiro emprego registrado) na Telem�tica Sistemas Inteligentes Ltda, tamb�m conhecida como Icatel, e respons�vel pela manuten��o de grande parte dos telefones p�blicos da cidade de S�o Paulo. Ali coloquei em pr�tica os conhecimentos pr�ticos adquiridos, consertando placas de circuito integrado dos extintos telefones p�blicos a ficha. Mas n�o tirei grandes proveitos: na �poca, minha �nica preocupa��o (lamentavelmente) era cumprir as horas do est�gio para concluir o ensino t�cnico.

Quando terminei o est�gio e o col�gio t�cnico, fui fazer cursinho (1996). Esta parte n�o se encaixa bem nem na forma��o acad�mica nem na profissional, mas foi um per�odo muito importante, pois consolidou os conhecimentos te�ricos que eu havia desenvolvido no ensino m�dio, al�m de corrigir as falhas de forma��o b�sica inerentes ao col�gio t�cnico (com muito tempo investido em disciplinas relativas � eletr�nica, as disciplinas b�sicas s�o prejudicadas). De fato, minha classifica��o no vestibulinho para o Liceu de Artes e Of�cios, na Escola T�cnica Estadudal de S�o Paulo e no Instituto Tecnol�gico de Osasco (ITO) foi apenas suficiente para me permitir entrar, e n�o fui aprovado no vestibulinho para a Escola T�cnica \emph{Federal} de S�o Paulo. Mas quando fiz o vestibular, cinco anos mais tarde, fui aprovado em 14\textordmasculine\ na USP para Bacharelado em F�sica, em 1\textordmasculine\ na UNESP para Ci�ncias da Computa��o e em 1\textordmasculine\ na classifica��o geral do FITO (Faculdade Instituto Tecnol�gico de Osasco). Tamb�m fui aprovado  para a UNICAMP (aparentemente em 51\textordmasculine\ na classifica��o geral, mas n�o tenho certeza desta informa��o). Resumindo, os anos de 1991 a 1995 foram de grande crescimento intelectual e profissional, e o cursinho � parte importante deste processo.

\section{Inicia��o cient�fica}

Logo que comecei a gradua��o, a inicia��o cient�fica no Laborat�rio de F�sica de Plasmas tornou-se minha �nica ocupa��o profissional (veja a se��o~\ref{sec:graduacao} para mais detalhes). Isto durou at� meados de 1998, quando fui aprovado, em 32\textordmasculine, num concurso p�blico para t�cnico banc�rio na Caixa Econ�mica Federal (CEF).

\section{Caixa Econ�mica Federal}
\label{sec:cef}

Na CEF voltei a desenvolver a habilidade de lidar com o p�blico: eu fui inicialmente designado para o setor de FGTS (Fundo de Garantia do Tempo de Servi�o), onde ocorriam os mais distintos e complexos problemas. Trabalhei tamb�m como caixa, com empr�stimos pessoais e estudant�s, com financiamentos de habita��o e com aplica��es, mas foi no FGTS que me destaquei e me especializei, criando procedimentos e mecanismos para otimizar o atendimento daquele setor.

Mas a contribui��o mais importante desse per�odo foi a possibilidade de eu comprar meu primeiro computador, o que iniciou a trajet�ria que percorro at� hoje (com sal�rio de bolsista do CNPq isto teria sido imposs�vel). Foi gra�as a ele que eu aprendi \foreign{hardware} de PC (lembro-me de ter lido um livro inteiro do Gabriel Torres, antes de fazer a compra), \LaTeX, Windows e Linux (na �poca em que se difundia fora do meio acad�mico), AutoCAD, CorelDraw, MathCAD, Mathematica, Matlab, Microcal Origin, Photoshop, 3D Studio e tantos outros (sem falar nos jogos e simuladores de voo). Este equipamento durou at� o final do meu mestrado, e lembro-me, com saudade, de t�-lo usado para escrever minha disserta��o.

Permaneci na Caixa Econ�mica Federal at� o come�o de 2001, quando ent�o fui contratado para desenvolver \foreign{software} na Eletropiezo Ind�stria e Com�rcio Ltda.

\section{Eletropiezo Ind�stria e Com�rcio Ltda}
\label{sec:eletropiezo}

Em abril de 2001, por indica��o de um colega da gradua��o, fui contratado pela Eletropiezo Ind�stria e Com�rcio Ltda, uma empresa que produz \foreign{software} para atendimento telef�nico (URA, de Unidade de Resposta Aud�vel). Foi neste meio que comecei a programar comercialmente, e passei a ter um tutor na �rea de programa��o de computadores: o colega e amigo Gerson de Souza Faria.

Aprendi a programar em T-REXX, uma linguagem propriet�ria da IBM usada para produzir URA, especificamente para o �nico projeto de URA IBM em Windows no Brasil, utilizando uma ferramenta chamada DirectTalk (hoje parte do pacote Websphere da IBM). Este trabalho foi desenvolvido para a Fidelity International Systems (FIS), que administra cart�es de in�meros bancos e agentes financeiros, como o Banco Ita�, Panamericano, e at� bancos menos conhecidos, como Rural, que ganhou notoriedade em 2006 com esc�ndalos de corrup��o.

Aprendi muitas t�cnicas novas de programa��o, os princ�pios da programa��o orientada a objetos, bem como a trabalhar em equipe e sob a press�o de prazos e responsabilidades: na gradua��o um erro custava nota; ali custava --- muito --- dinheiro.

Deixei a empresa no in�cio de 2002, para come�ar o mestrado, mas continuei dando suporte t�cnico significativo at� muito recentemente, pois acabei me tornando um dos poucos profissionais capacitados para este trabalho no Brasil. Para isso precisei abrir uma empresa de desenvolvimento de \foreign{software}, a Cagnotto \& Pagnossin Servi�os de Inform�tica Ltda.

O projeto foi um sucesso para todos os envolvidos e manteve-se ativo at� dezembro de 2010 (se voc�, leitor, tem um cart�o de cr�dito, muito provavelmente j� foi atendido por essa URA), quando ent�o foi integralmente substitu�do por uma vers�o mais recente (da qual eu n�o participei).

\section{Cagnotto \& Pagnossin Ltda}

Esta � a empresa da qual sou dono. Ela foi inicialmente aberta, em novembro de 2005, para a presta��o de servi�o de desenvolvimento de \foreign{software} e suporte t�cnico das URAs da FIS. Mas desde ent�o esta empresa tem prestado servi�o para outros clientes, como o Instituto de Pesquisas Eldorado (projeto Aulas Interativas, mais a frente), a Coordenadoria de Tecnologia da Informa��o (CTI), a Funda��o de Apoio � USP (FUSP) e o pr�prio Instituto de F�sica da USP.

\section{Centro de Ensino e Pesquisa Aplicada}
\label{sec:cepa}

Minhas atividades no Centro de Ensino e Pesquisa Aplicada (CEPA) representam a conflu�ncia das minhas trajet�rias acad�mica e profissional, e certamente consistem nas minhas mais relevante contribui��es para a sociedade e para a USP.

No final de 2007 eu estava bastante descontente com os resultados obtidos no doutorado e desiludido com a morosidade da pesquisa experimental (ao menos na �rea em que eu atuei). E o prosseguimento padr�o seria conseguir uma bolsa de p�s-doutoramento, uma ideia que n�o me agradava.

Nesta ocasi�o, e por interm�dio da \profa\ Euzi, conheci o Prof. Dr. Gil da Costa Marques, criador e respons�vel por um grupo do Departamento de F�sica Experimental do IFUSP dedicado � cria��o de material did�tico, o CEPA. Ele precisava de um programador para desenvolver \foreign{applets} Java de simula��es de fen�menos f�sicos, como parte do projeto TIDIA-Ae \cite{tidia}, e minha forma��o acad�mica e profissional fazia de mim a pessoa certa para o trabalho.

Para mim era uma conjun��o favor�vel: congregar F�sica e desenvolvimento de \foreign{software}, as duas principais �reas nas quais eu vinha investindo h� dez anos. Al�m disso, a confec��o da minha disserta��o de mestrado e da minha tese de doutorado desenvolveram em mim a capacidade de criar ilustra��es, anima��es e simula��es agrad�veis aos olhos (\edital{ferramentas de produ��o gr�fica}); e mais importante, a capacidade de simplificar a apresenta��o de ideias complexas.

E assim comecei a trabalhar no CEPA, com bolsa de capacita��o t�cnica n�vel TT-4 da FAPESP, em dezembro de 2007. Nos primeiros seis meses eu trabalhei sozinho, pois era o �nico programador de simula��es da equipe, e desenvolvi \foreign{applets} sobre campos vetoriais, integrais de linha, lan�amento bal�stico com resist�ncia do ar, �ngulos de Euler (o primeiro que envolvia o uso de programa��o tridimensional), entre v�rios outros \cite{applets}.

No in�cio esses \foreign{applets} distinguiam-se dos demais, encontrados na Internet, apenas pelo \edital{\foreign{design} da intera��o} (ou, de forma mais ampla, a \edital{experi�ncia do usu�rio}), embora ainda sutilmente. Mas esta preocupa��o guiou meu trabalho com \foreign{applets} nos meses seguintes, onde procurei desenvolver m�todos para trabalhar conjuntamente com artistas, que ent�o ficariam respons�veis pela parte visual. A ideia era que um recurso \emph{did�tico} precisava n�o apenas passar o conceito a que se propunha, mas t�o importante quanto isso, precisava tamb�m cativar o usu�rio. Os resultados deste trabalho podem ser obtidos no meu perfil na \edital{comunidade social} Stoa \cite{irpagnossin-stoa}.

\subsection{O curso \textsl{Usando \LaTeX; pensando \TeX}}
\label{sec:latex}

Ainda no primeiro semestre de 2008, o Prof. Gil pediu que eu montasse um curso sobre \LaTeX, um \edital{sistema de produ��o de documentos} que eu havia aprendido a utilizar na gradua��o, para os relat�rios de laborat�rio. Desta encomenda surgiu o curso ``Usando \LaTeX; pensando \TeX'', que foi oferecido para a comunidade USP atrav�s da Coordenadoria de Tecnologia da Informa��o (CTI).

O curso foi concebido inicialmente para ser semi-presencial, com 20 horas de dura��o. O enfoque dele era totalmente pr�tico (usando \LaTeX), mas explorava profundamente os conceitos fundamentais do sistema (pensando \TeX). E por ser semi-presencial, todo o conte�do do curso, como tutoriais, apresenta��es, atividades pr�ticas, exerc�cios e li��es (muitos deles com avalia��o autom�tica e um extenso e cuidadoso sistema de \foreign{feedbacks}), foi disponibilizado no sistema de gerenciamento de cursos Moodle, que aprendi a usar num curso oferecido pela CTI. N�o obstante isso, a parte n�o-presencial valia-se tamb�m das mais modernas ferramentas de aprendizado colaborativo e da \edital{web 2.0}, como f�runs, chats e \edital{wikis}. J� a parte presencial do curso ocorria na sala multimeios do IFUSP, que contava com 25 \foreign{notebooks} e uma lousa eletr�nica, equipamentos que foram realmente utilizados no curso.

A primeira turma oficial foi aberta no segundo semestre de 2008, tendo eu como professor (um trabalho \foreign{pro bono}), minha colega Juliana Giordano como tutora e meu colega Marcelo Alves como \foreign{designer} instrucional\footnote{A primeira turma de fato (experimental) ocorreu em julho e agosto de 2008, e contou com a presen�a do Prof. Gil e de funcion�rios do CEPA, da CTI e do IFUSP que precisavam daqueles conhecimentos, principalmente para auxiliar os professores na escrita de seus artigos cient�ficos.}. E foi um grande sucesso, mostrando que havia de fato interesse por um curso assim na USP: em menos de duas horas ap�s a abertura das inscri��es, na comunidade Stoa, as 25 vagas j� estavam completas; e antes do final daquele dia, j� havia mais de 150 inscritos.

Mas o curso exigia muito dos alunos, pois concentrava-se em atividades: metade de \emph{toda} aula presencial era composta por exerc�cios. E no Moodle existiam in�meras atividades e exerc�cios para serem feitas, com graus de complexidade crescentes. De fato, apenas 11 pessoas concluiram a primeira turma, e cada uma recebeu um diploma endossado pelo Prof. Gil, ent�o coordenador da CTI.

Em seguida o curso foi reformulado e expandido para 24 horas, e uma nova turma foi oferecida no primeiro semestre de 2009 (eu novamente como professor). Delas, 7 chegaram ao final.

Este curso foi uma das minhas mais significativas produ��es no CEPA e ele continua dispon�vel atrav�s da Internet \cite{curso-LaTeX}, mas nenhuma outra turma foi oferecida (ainda h� procura), pois a proposta inicial era que se tornasse um curso a dist�ncia. Ademais, um novo projeto entrava em cena, que requereria toda a minha aten��o: o projeto Aulas Interativas.

\subsection{O projeto Aulas Interativas}
\label{sec:aulas-interativas}

Ainda no primeiro semestre de 2009, o CEPA foi procurado pela \profa\ Maria Alice Carraturi Pereira, ent�o Assessora de Tecnologia Educacional da Secretaria de Estado da Educa��o (SEE), para um projeto em parceria com a Dell Computadores do Brasil. A proposta era instalar uma lousa eletr�nica em cada uma das 26 escolas p�blicas da regi�o de Hortol�ndia, interior de S�o Paulo, e caberia ao CEPA produzir os conte�dos interativos para as lousas, para as disciplinas de L�ngua Portuguesa e Matem�tica da 6\textordfeminine\ s�rie do Ensino Fundamental e do 1\textordmasculine\ ano do Ensino M�dio.

Na verdade, o CEPA fora convidado a apresentar uma proposta de aula interativa, com lousa eletr�nica (concorr�amos com outras empresas, como Clickideia, Klick educa��o e Funda��o Conesul). E coube a mim montar essa aula e apresent�-la\footnote{Cabe mencionar que as habilidade desenvolvidas na cria��o do curso de \LaTeX\ contribuiram bastante, especialmente no que se referia a falar em p�blico, uma tarefa que eu passei a n�o ter dificuldade.} para membros da SEE e da Dell. O t�pico, escolhido pela SEE, era ``as rela��es m�tricas do tri�ngulo-ret�ngulo''. Ao montar a aula, tomei o cuidado de partir de conceitos cotidianos (algo que em pedagogia se chama construcionismo), e usando a lousa eletr�nica, mostrei como a dedu��o das rela��es m�tricas do tri�ngulo-ret�ngulo tornava-se simples quando valendo-se de um \foreign{software} que eu produzi especialmente para esta apresenta��o. A proposta agradou, pois o CEPA foi escolhido para o trabalho (e eu fui convidado a reapresentar esta aula in�meras outras vezes).

Curiosamente, conv�m mencionar que este projeto quase n�o foi concretizado devido �s falhas encontradas nos cadernos de Geografia, naquele ano, e que causaram a queda da ent�o Secret�ria da Educa��o.

Assim o projeto come�ou, pouco antes do segundo semestre de 2009, com enfoque voltado para a produ��o de \foreign{softwares} interativos para lousas eletr�nicas. E eu passei a liderar uma pequena equipe de programadores (com estagi�rios da USP, inclusive), bem como a coordenar a produ��o desse material com a equipe de arte do CEPA, at� ent�o ausentes na cria��o desse tipo de conte�do (lembre-se: at� ent�o eu era o �nico programador do CEPA e trabalhava sozinho). Mais especificamente, eu me tornei respons�vel pela produ��o de todo o material de Matem�tica, e embora oficialmente eu n�o fora indicado para propor os conte�dos interativos, com base nos cadernos desenvolvidos pela SEE, pois minha forma��o n�o era a de Matem�tico, eu muito influenciei esse material, num trabalho conjunto com os professores indicados pela SEE (e que haviam criado os cadernos) e com os Professores Coordenadores das Oficinas Pedag�gicas (PCOP).

Cheguei tamb�m a treinar professores da rede p�blica, usu�rios das lousas e do material por n�s criado, em uma aula tem�tica criada para a abertura oficial do projeto, na SEE, em novembro de 2009. Esta apresenta��o teve as presen�as ilustres do ent�o Secret�rio de Educa��o, Paulo Renato Souza, e do fundador da Dell, Michael Dell.

Atrav�s deste projeto eu aprendi novas tecnologias como Flash, ActionScript e lousa eletr�nica; aprendi a trabalhar com pessoas de outras �reas, como os professores da SEE e os ilustradores do CEPA; desenvolvi em mim a capacidade de liderar uma equipe, e coorden�-la para atingir um objetivo (dentro do prazo, sempre que poss�vel); e por que n�o dizer: tive a oportunidade de conhecer pessoalmente as prec�rias situa��es em que os professores da regi�o de Sumar� bravamente exercem sua profiss�o, e apesar disso, ainda assim observar o brilho nos olhos de cada aluno(a) com as novidades que lev�vamos para eles.

Para o CEPA, foi a oportunidade de evoluir --- muito --- em termos de profissionalismo, e prepar�-lo para o que estava por vir: os projetos Univesp e Redefor. A participa��o do CEPA, na produ��o de material, terminou em dezembro de 2010, mas o projeto continua em funcionamento, utilizando o material produzido por n�s.

Em tempo, � importante observar que a escolha pela tecnologia Flash, feita por mim (eu era o �nico programador do CEPA), foi crucial para que o projeto fosse cumprido. Que a verdade seja dita: de um lado, o governo de Jos� Serra impusera que o projeto deveria estar pronto at� o final de 2010 (�poca de elei��o); do outro, a SEE exigira um ano letivo completo de material produzido. Esses eram os termos para a Dell. Para o CEPA significava um prazo irris�rio. Tivesse eu escolhido utilizar Java, n�s n�o ter�amos chegado ao final do projeto, pois a produtividade dele � bem menor que a do Flash.

\subsection{Os projetos Univesp e Redefor}

Ap�s a assinatura do Governador Jos� Serra para a cria��o da Univesp (Universidade Virtual do Estado de S�o Paulo), em mar�o de 2010\footnote{A aprova��o deste projeto vinha sendo adiada h� pelo menos um ano por press�es de setores da sociedade contr�rios � educa��o a dist�ncia.}, o CEPA passou a produzir parte do material que seria disponibilizado aos alunos atrav�s do sistema Moodle (usado no curso de \LaTeX). A primeira turma do primeiro curso, Licenciatura em Ci�ncias, teve in�cio no segundo semestre de 2010.

A partir da� eu adquiri a tarefa dupla de garantir a produ��o de materiais (\foreign{softwares} interativos) tanto para a Univesp quanto para o projeto Aulas Interativas. Al�m disso, eu oficialmente fa�o parte do projeto Univesp como Educador II (ou professor de atividades) das disciplinas de Din�mica dos Corpos e Eletromagnetismo, respons�vel pela cria��o de atividades \foreign{online}; no Moodle.

� importante mencionar tamb�m que, com a Univesp, os \foreign{softwares} que desenvolvemos passaram a ser desenvolvidos para a \edital{Internet} (isto �, levando-se em conta diferentes \edital{navegadores} e \edital{sistemas operacionais}), uma vertente que n�o era desenvolvida no projeto Aulas Interativas.

Al�m da produ��o, grande parte do meu esfor�o atual � dedicado � otimiza��o dos processos internos do CEPA. Um exemplo disso � a implanta��o (ainda parcial) do sistema de desenvolvimento �gil de projetos chamado Scrum, que eu trouxe para o CEPA no final de 2010. Outro � a utiliza��o do padr�o \edital{SCORM} (\foreign{Sharable Content Object Reference Model}) nos \foreign{softwares} que desenvolvemos (Objetos de Aprendizagem, na terminologia \foreign{e-learning}), tamb�m introduzido por mim em meados de 2010, e que eu j� vinha estudando um ano antes. O SCORM 1.2 � o padr�o de fato da ind�stria de \foreign{e-learning} no exterior, mas ainda � muito pouco difundido no Brasil. No CEPA ele � empregado parcialmente, e j� rendeu uma bolsa de treinamento t�cnico para um dos integrantes da minha equipe (oficialmente, ele � um aluno do Prof. Dr. Ewout ter Haar).
 
A equipe que hoje lidero e coordeno � composta por 7 pessoas, al�m de mim: um ilustrador, tr�s programadores e tr�s estagi�rios (de programa��o). No entanto, eu participo de praticamente todas as decis�es que envolvem o CEPA. Al�m disso, tenho tamb�m participado do desenvolvimento do novo leiaute do Moodle, juntamente com a equipe de Apoio T�cnico e Pedag�gico (parte do CEPA), liderada pelo Prof. Dr. Ewout ter Haar, e a \foreign{designer} instrucional \profa\ Vani Kenski. Auxilio tamb�m a consultora Ely Joana Beloto na modelagem do processo de cria��o dos cursos, particularmente no que concerne o sistema de gerenciamento de projetos que utilizamos, o Redmine.

O CEPA e, particularmente, a equipe de cria��o de objetos de aprendizagem, produzem material para outro grande projeto, tamb�m em parceria com a Secretaria de Estado da Educa��o: a Redefor, ou Rede S�o Paulo de Forma��o Docente. Contudo, minha participa��o neste projeto ainda � pequena, mas vem crescendo nos �ltimos meses com a divulga��o dos objetos de aprendizagem criados para a Univesp.

\subsection{Resumo}

Entrar no CEPA foi a reuni�o de duas voca��es, F�sica/Matem�tica e programa��o, e hoje percebo que estou no lugar certo. Contudo, ainda h� uma falha que n�o tive a oportunidade de consertar: o CEPA � hoje um centro de produ��o de material did�tico, n�o de pesquisa. E isto � inaceit�vel, pois temos pessoal e material para realiz�-la. O que nos falta � um docente USP dedicado a isso, que possa orientar essa pesquisa e pleitear financiamentos em �rg�os de fomento. De fato, j� perdemos algumas boas oportunidades por conta disso (o Prof. Gil, coordenador do grupo, est� ocupado com quest�es maiores, e o Prof. Ewout est� envolvido com outras pesquisas).

De todo modo, em breve o CEPA produzir� tamb�m conhecimento de ponta...
	\chapter{Perspectivas}

Eu tive a oportunidade de experimentar os meios acad�mico e corporativo, e a felicidade de trilhar o ``caminho do meio'', a despeito de todas as d�vidas, que sempre estiveram presentes. Muito foi realizado, especialmente no CEPA, e em grande parte sem a recompensa financeira que eu poderia obter no meio corporativo. Mas esta � uma escolha que fa�o lucidamente, pois as minhas maiores recompensas s�o pessoais, e acima de tudo minha inten��o � construir algo de que me orgulhe e que contribua para a nossa sociedade. Eu tenho conseguido seguir este caminho, bem ou mal, com a ajuda de todas as pessoas que encontrei nessa jornada. Mas ela n�o tem fim...

Cria��o e produ��o de objetos de aprendizagem, e pesquisa sobre eles. Esses s�o os tr�s ingredientes que preciso. Atualmente tenho os dois primeiros, e j� come�o a lutar para conseguir o terceiro. E � justamente onde este cargo de docente pode auxiliar. Dito de outra forma, ter entrado no CEPA foi o primeiro passo; ser docente da USP � um poss�vel segundo (mas certamente n�o �nico).

Independentemente disso, meu trabalho no CEPA continua, e para o pr�ximo semestre (at� o final do ano) tenho algumas metas bem claras:

\begin{compactitem}
	\item Finalizar a implanta��o do Scrum no CEPA;
	\item Buscar \foreign{feedback} dos alunos, tutores e educadores quanto aos objetos de aprendizagem disponibilizados;
	\item Difundir os objetos de aprendizagem entre os professores autores, educadores e tutores (muitos deles sequer sabem que eles existem, e outros n�o imaginam que existe uma equipe capaz de realizar suas ideias);
	\item Difundir o padr�o SCORM entre os professores autores, educadores e tutores;
	\item Submeter dois ou tr�s trabalhos, em parceria com colegas acad�micos do CEPA, para os pr�ximos congressos da ABED (Associa��o Brasileira de Ensino a Dist�ncia), baseados nas experi�ncias nos projetos Aulas Interativas, Univesp e Redefor.
\end{compactitem}

	
	% Apêndices
	\appendix
	
	\section{Adendo}
\label{sec:adendo}

	Caro professor, este adendo apresenta brevemente alguns conteúdos educacionais produzidos no CEPA, por mim ou sob a minha coordenação. Optei por não inserir este material no memorial propriamente dito pois ele requereria explicações que iriam além do propósito dele. Por outro lado, creio que apenas pelo texto seja muito difícil imaginar o que estamos fazendo. Daí este adendo: seu intuito é dar uma ideia do que sejam esses \foreign{softwares} educacionais.
	
	\begin{figure}
		\centering
		\begin{minipage}[b]{0.46\textwidth}
			\includegraphics[width=\textwidth]{images/campo-vetorial.jpg}
			\caption{\footnotesize primeiro \foreign{applet} Java desenvolvido para o CEPA, em dezembro de 2007. Este recurso é expositivo, mas permite ao usuário interagir de modo a comparar seus cálculos com aqueles apresentados pelo \foreign{software}.}
			\label{fig:campo}
		\end{minipage}\hfill
		\begin{minipage}[b]{0.46\textwidth}
			\includegraphics[width=\textwidth]{images/triangulo-retangulo.jpg}
			\caption{\footnotesize \foreign{software} desenvolvido para a proposta de aula interativa do CEPA, em maio de 2009. Este recurso é expositivo e colaborativo, pois foi feito para ser executado em lousas eletrônicas.}
			\label{fig:rel-metr}
		\end{minipage}
	\end{figure}
	
	A figura~\ref{fig:campo} ilustra o primeiro \foreign{applet} Java que desenvolvi quando ingressei no CEPA, em dezembro de 2007. Seu propósito é explorar os conceitos de integral de linha e de campos vetoriais conservativos ($\vec u = -\nabla f$) e não-conservativos. Essencialmente, o \foreign{software} exibe o valor da integral de linha, calculada numericamente.
	
	Inicialmente, o \foreign{software} disponibiliza um campo vetorial (conservativo) e um percurso fáceis de manipular, para que o usuário possa fazer os cálculos por conta própria, tanto da integral de linha quanto de $f(B) - f(A)$, oriundo do teorema fundamental do cálculo. Em seguida, ele é orientado a modificar o percurso, sem mover os pontos $A$ e $B$, e a perceber que a integral não muda (consequência do campo ser conservativo). Diferentemente, quando o usuário escolhe um campo não-conservativo, a integral altera-se com qualquer mudança no percurso. Este recurso está disponível \href{http://cepa.if.usp.br/old/files/simulation/javaapplet/PathIntegralAtContainer.html}{aqui}.
	
	A figura~\ref{fig:rel-metr} ilustra o aplicativo que foi desenvolvido especialmente para a proposta de aula interativa com lousa eletrônica do CEPA, para o projeto Aulas Interativas. Seu intuito é \emph{auxiliar o professor} a deduzir as relações métricas do triângulo-retângulo: o professor arrasta de dentro do triângulo maior (no topo) os dois triângulos-retângulos internos, e pode rotacioná-los e refletí-los convenientemente, de modo que fique evidente a relação de semelhança entre eles. Desta forma, o professor pode deduzir muito facilmente as relações, sem os embaraços da abordagem tradicional. Por exemplo, a relação $c^2 = a \cdot m$ pode ser deduzida observando-se os lados $a$ e $c$ do triângulo superior e os lados $c$ e $m$ daquele à esquerda: por semelhança, $c/m = a/c$, o que resulta na relação métrica.
	
	Pelo que observei nas apresentações desta aula, a simplicidade desta dedução foi o que realmente cativou os professores da Secretaria de Estado da Educação, e creio que tenha contribuido decisivamente para a participação do CEPA no projeto.
	
	\begin{figure}
		\centering
		\begin{minipage}[t]{0.46\textwidth}
			\includegraphics[width=\textwidth]{images/navigation.jpg}
			\caption{\footnotesize recurso desenvolvido para a disciplina de Matemática do 6\textordmasculine\ do Ensino Fundamental (projeto Aulas Interativas). Como o anterior, é expositivo e colaborativo, mas também permite avaliação do aluno, através da comparação entre os trajetos azul e vermelho.}
			\label{fig:nav}
		\end{minipage}\hfill
		\begin{minipage}[t]{0.46\textwidth}
			\includegraphics[width=\textwidth]{images/polar.jpg}
			\caption{\footnotesize este \foreign{software} foi desenvolvido para a Univesp, e traz uma abordagem mais próxima dos jogos eletrônicos. Ele também permite avaliação e foi feito para aulas a distância.}
			\label{fig:polar}
		\end{minipage}
	\end{figure}
	
	
	A figura~\ref{fig:nav} representa uma das atividades interativas que produzimos para o projeto Aulas Interativas, para a disciplina de Matemática do 6\textordmasculine\ ano do Ensino Fundamental. O intuito dela é permitir ao professor expor e exercitar, de maneira lúdica, conceitos como pontos cardeais, escala, medidas de ângulo e de distância, e até uma introdução à programação de computadores.
	
	Por exemplo, o professor pode desenhar uma trajetória (azul) arbitrária para o navio e, em seguida, pedir que os alunos representem aquela trajetória em termos de uma sequência de comandos \emph{mover}, \emph{girar} e \emph{repetir}. Isto também é feito no \foreign{software} (na lousa eletrônica), e é representado pelo script na base da imagem. Para executar esta tarefa a contento, os alunos devem utilizar a régua e o transferidor disponibilizados no \foreign{software}, bem como a escala do mapa. E para conferir a resposta, o professor pode habilitar a visualização desse caminho, em vermelho. Alternativamente, o professor pode escrever o script (rota em vermelho) e pedir que os alunos a desenhem, em azul. Observe ainda que o professor pode executar a animação do navio percorrendo a trajetória azul ou a vermelha passo-a-passo, e assim discutir cada instrução dada no script.
	
	Particularmente com relação a este \foreign{software}, outra possibilidade que chamou a atenção dos Professores Coordenadores das Oficinas Pedagógicas foi a possibilidade de ilustrar como é que o computador desenha uma curva: basta colocar uma instrução do tipo \emph{repetir 10 vezes os comandos: mover \unit{10}{\kilo\metre} e girar \unit{10}{\degree}}.
	
	Finalmente, a figura~\ref{fig:polar} ilustra outro recurso, este feito para a Univesp. Seu objetivo é permitir ao usuário \emph{familiarizar-se} com os sistema de coordenadas polar. Ao pressionar o botão \emph{ok} o alvo é aleatoriamente posicionado na tela, e cabe ao usuário acertá-lo com um dardo. Para lançar o dardo o usuário deve informar quais são as coordenadas, $r$ e $\theta$, do alvo. E quanto mais próximas da resposta, mais próximo do alvo o dardo chega e, por conseguinte, mais pontos ele faz. Há uma régua e um transferidor à disposição do usuário (embaixo, à esquerda), de modo que ele pode medir as coordenadas com precisão. No entanto, há um tempo limite para dar a resposta, e quanto mais rápido ele resolve a questão, mais pontos ele faz. Assim, para maximizar sua pontuação o usuário acaba concluindo, depois de algumas tentativas e erros, que é melhor \emph{estimar} a resposta. E deste modo espera-se que ele se familizarize com as coordenadas polares. Este recurso está disponível em \href{http://midia.atp.usp.br/atividades-interativas/AI-0001/}{aqui}.
	
	\chapter{Dados pessoais e produ��es relevantes}
\label{cap:resumo}

%--------------------------------------------------------------------------
\section*{Dados pessoais e de contato}
\addcontentsline{toc}{section}{Dados pessoais e de contato}

\begin{compactdesc}
	\item[Nome:] Ivan Ramos Pagnossin
	\item[RG:] 15.420.406-7
	\item[CPF:] 179.905.018-13
	\item[Celular:] (11) 6434-4513
	\item[Telefone residencial:] (11) 3686-7583
	\item[Endere�o residencial:] Rua Rio S�o Francisco, 287 --- Osasco/SP --- CEP 06236-070
	\item[Telefone comercial:] (11) 3091-6695 ou 3091-6709
	\item[Endere�o comercial:] CEPA (Centro de Ensino e Pesquisa Aplicada), Rua do Mat�o, Travessa R, 187
Edif�cio Van de Graaf --- Cidade Universit�ria-USP --- S�o Paulo/SP --- CEP 05508-090
	\item[e-mail:] ivan.pagnossin@gmail.com
\end{compactdesc}

\begin{tabbing}
Meus curr�culos na Internet: \=\url{http://goo.gl/lKz1u}\`Curr�culo Lattes\\
														 \>\url{http://goo.gl/jWcJD}\`LinkedIn\\
														 \>\url{http://goo.gl/N1MUC}\`curriculum.com.br
\end{tabbing}

%--------------------------------------------------------------------------
\section*{Forma��o acad�mica e titula��o}
\addcontentsline{toc}{section}{Forma��o acad�mica e titula��o}

\begin{compactdesc}
	\item[Ensino Fundamental:] Instituto S�o Pio X, de 1982 a 1990.
	\item[Ensino M�dio:] Liceu de Artes e Of�cios de S�o Paulo, de 1991 a 1995.
	\item[Ensino M�dio Profissionalizante:] Liceu de Artes e Of�cios de S�o Paulo, de 1991 a 1995. T�tulo de T�cnico em Eletr�nica.
	\item[Gradua��o:] IFUSP, de 1997 a 2002. T�tulo de Bacharel em F�sica B�sica.
	\item[Inicia��o cient�fica:] Laborat�rio de F�sica de Plasmas do IFUSP, de 1997 a 1998.
	\item[Mestrado:] IFUSP, de 2002 a 2004. T�tulo de Mestre em Ci�ncias.
	\item[Doutorado:] IFUSP, de 2005 a 2007. T�tulo de Doutor em Ci�ncias.
\end{compactdesc}

%--------------------------------------------------------------------------
\section*{Forma��o complementar}
\addcontentsline{toc}{section}{Forma��o complementar}

\begin{compactitem}
	\item \textsl{Curso Profissional de Datilografia}, Tecla Escola de Datilografia, 1989.
	\item \textsl{Cursinho para o exame vestibulinho}, Centro Educacional Desafio, 1990.
	\item \textsl{Desenho t�cnico}, Liceu de Artes e Of�cios de S�o Paulo, 1991.
	\item \textsl{Desenho art�stico}, Liceu de Artes e Of�cios de S�o Paulo, 1992.
	\item \textsl{Reconhecimento do c�u}, Escola Municipal de Astrof�sica Planet�rio Municipal, 1992 (15 horas).
	\item \textsl{T�picos de Astronomia: movimentos da Terra}, Escola Municipal de Astrof�sica Planet�rio Municipal, 1993 (10 horas).
	\item \textsl{Fundamentos de Astrof�sica II --- Evolu��o estelar}, Escola Municipal de Astrof�sica Planet�rio Municipal, 1993 (30 horas).
	\item \textsl{Cursinho para o exame vestibular}, Etapa, 1996.
	\item \textsl{Alem�o}, Faculdade de Filosofia, Letras e Ci�ncias Humanas, 2003 (ouvinte do primeiro semestre do curso de Letras).
	\item \textsl{Ingl�s}, Top English, 2002.
	\item \textsl{Curso de voo a vela}, Aeroclube Polit�cnico de Planadores, 2004--2009 (40 horas de voo).
	\item \textsl{Curso preparat�rio para o TOEFL}, Faculdade de Filosofia, Letras e Ci�ncias Humanas da USP, 2006.
	\item \textsl{Franc�s}, Franc�s em casa, 2007 (50 horas).
	\item \textsl{Moodle na USP}, CEPA/CTI, 2008.
	\item \textsl{Desenvolvimento Web com HTML, CSS e JavaScript}, Caelum Ensino e Inova��o, 2010 (20 horas).
	\item \textsl{Leader Training I}, Arita Treinamentos, 2010 (35 horas).
	\item \textsl{Gerenciamento �gil de projetos com Scrum}, Caelum Ensino e Inova��o, 2010 (20 horas).
	\item \textsl{Brigadista para combate ao inc�ndio}, Centro de Treinamento da Roch�cara Ecofire, 2010 (12 horas).
	\item \textsl{Leader Training II}, Arita Treinamntos, 2011 (35 horas).
\end{compactitem}

%--------------------------------------------------------------------------
\section*{Desenvolvimento de material did�tico ou instrucional}
\addcontentsline{toc}{section}{Desenvolvimento de material did�tico ou instrucional}

\begin{compactitem}
	\item 16 Objetos de Aprendizagem para \foreign{web} (F�sica), no �mbito do projeto TIDIA-Ae.
	\item 104 Objetos de Aprendizagem para lousa eletr�nica (Matem�tica e L�ngua Portuguesa), no �mbito do projeto Aulas Interativas, sendo 31 produzidas por mim e 73 sob a minha coordena��o.
	\item 118 Objetos de Aprendizagem para \foreign{web} (F�sica e Matem�tica), no �mbito dos projetos Univesp e Redefor, sendo 32 produzidas por mim e 86 sob a minha coordena��o.
\end{compactitem}

%--------------------------------------------------------------------------
\section*{Produ��o bibliogr�fica}
\addcontentsline{toc}{section}{Produ��o bibliogr�fica}

\begin{compactitem}

	\item Disserta��es e teses
	\begin{compactitem}
		\item \textsl{Propriedades de transporte el�trico de gases bidimensionais de el�trons nas proximidades de pontos-qu�nticos de InAs}, Disserta��o de Mestrado, IFUSP (2004).
		\item \textsl{Pontos-qu�nticos: fotodetectores, localiza��o-fraca e estados de borda contra-rotativos}, Tese de Doutorado, IFUSP (2008).
	\end{compactitem}
	
	\item Artigos completos publicados em peri�dicos
	\begin{compactitem}
		\item I. R. Pagnossin, E. C. F. da Silva, A. A. Quivy, S. Martini e C. S. Sergio, \textsl{The quantum mobility of a two-dimensional electron gas in selectively doped GaAs/InGaAs quantum wells with embedded quantum dots}, J. Appl. Phys. \textbf{97}, 113709 (2005).
		\item I. R. Pagnossin, A. K. Meikap, A. A. Quivy, G. M. Gusev, \textsl{Electron dephasing scattering rate in two-dimensional GaAs/InGaAs heterostructures with embedded InAs quantum dots}, J. Appl. Phys. \textbf{104}, 073723 (2008).
		\item I. R. Pagnossin, A. K. Meikap, T. E. Lamas, G. M. Gusev, J. C. Portal, \textsl{Anomalous dephasing scattering rate of two-dimensional electrons in double quantum well structures}, Phys. Rev. B, Condensed Matter and Materials Physics \textbf{78}, 115311 (2008). 
	\end{compactitem}
		
	\item Trabalhos completos publicados em anais de congressos
	\begin{compactitem}
		\item I. R. Pagnossin, E. C. F. da Silva, A. A. Quivy, S. Martini, C. S. Sergio, \textsl{The influence of strain fields around InAs quantum dots on the transport properties of a two-dimensional electron gas confined in GaAs/InGaAs wells}, no 12th Brazilian Workshop on Semiconductor Physics, 2005, S�o Jos� dos Campos. Brazilian Journal of Physics, 2005.
		\item I. R. Pagnossin, E. C. F. da Silva, A. A. Quivy, S. Martini, C. S. Sergio, \textsl{Scattering processes on a quasi-two-dimensional electron gas in GaAs/InGaAs selectively doped quantum wells with embedded quantum dots}, no 12th Brazilian Workshop on Semiconductor Physics, 2005, S�o Jos� dos Campos. Brazilian Journal of Physics, 2005.
		\item I. R. Pagnossin, G. M. Gusev, A. C. Seabra, A. A. Quivy, T. E. Lamas, J.-C. Portal, \textsl{Quantum Hall effect in bilayer system with array of antidots}, no 28th International Conference on the Physics of Semiconductors (ICPS-28), 2006, Viena. 28th International Conference on the Physics of Semiconductors (ICPS-28), 2006. p.~96.
		\item I. R. Pagnossin, G. M. Gusev, N. M. Sotomayor, A. C. Seabra, A. A. Quivy, T. E. Lamas, J.-C. Portal, \textsl{Quantum Hall effect in bilayer system with array of antidots}, no 28th International Conference on the Physics of Semiconductors (ICPS-28) --- Viena/Austria, 2006, Viena. Physics of Semiconductors, 28th International Conference, 2006. pp.~677--678.
	\end{compactitem}
	
	\item Resumos publicados em anais de congressos
	\begin{compactitem}
		\item I. R. Pagnossin, E. C. F. da Silva, A. A. Quivy, J. R. Leite, S. Martini, C. S. Sergio, \textsl{The influence of an InAs layer on the quantum mobility of a two-dimensoinal electron gas in GaAs/InGaAs selectively doped quantum wells}, no XXVII Encontro Nacional da Mat�ria Condensada, 2004, Po�os de Caldas. XXVII ENFMC, 2004. v. 1. pp.~399--399.
		\item I. R. Pagnossin, A. K. Meikap, A. A. Quivy, G. M. Gusev, \textsl{Weak localization and interaction effects in GaAs/InGaAs heterostructures with nearby InAs quantum-dots}, no 13th Brazilian Workshop on Semiconductor Physics --- S�o Paulo/SP, 2007, S�o Paulo. 13th Brazilian Workshop on Semiconductor Physics --- S�o Paulo/SP, 2007.
		\item I. R. Pagnossin, G. M. Gusev, A. C. Seabra, A. A. Quivy, T. E. Lamas, J.-C. Portal, \textsl{Quantum Hall effect in bilayer system with array of antidots}, no 28th International Conference on the Physics of Semiconductors (ICPS-28) --- Viena/Austria, 2006, Viena. 28th International Conference on the Physics of Semiconductors (ICPS-28) --- Viena/Austria, 2006.	
	\end{compactitem}
\end{compactitem}

%--------------------------------------------------------------------------
\section*{Relat�rios t�cnicos}
\addcontentsline{toc}{section}{Relat�rios t�cnicos}

\begin{compactitem}
	\item 12 relat�rios de atividade, para a FAPESP, referentes ao trabalho de desenvolvimento de \foreign{softwares} educacionais utilizando Java, no �mbito do projeto TIDIA-Ae, durante o per�odo de dezembro de 2007 a janeiro de 2009. Esses relat�rios podem ser obtidos atrav�s de \cite{irpagnossin-stoa}.
	\item 7 relat�rios mensais de atividade, para o Instituto de Pesquisa Eldorado, referentes ao trabalho de produ��o de \foreign{softwares} educacionais para o projeto Aulas Interativas, durante o per�odo de dezembro de 2009 a junho de 2010.
\end{compactitem}

%--------------------------------------------------------------------------
\section*{Bolsas e aux�lios}
\addcontentsline{toc}{section}{Bolsas e aux�lios}

\begin{compactitem}
	\item Ensino M�dio Profissionalizante, Liceu de Artes e Of�cios de S�o Paulo, 1991--1995.
	\item Inicia��o cient�fica, CNPq, 1997.
	\item Inicia��o cient�fica, CNPq, 1998.
	\item Mestrado, FAPESP, 2002--2004.
	\item Doutorado, FAPESP, 2005--2007.
	\item Capacita��o T�cnica, n�vel TT-4, FAPESP, 2008.
	\item Capacita��o T�cnica, n�vel TT-5, FUSP, 2010.
\end{compactitem}

%--------------------------------------------------------------------------
\section*{Aux�lios para viagens ao exterior}
\addcontentsline{toc}{section}{Aux�lios para viagens ao exterior}

\begin{compactitem}
	\item Grenoble High Magnetic Field Laboratory, Grenoble, Fran�a (mar�o a maio de 2007). Est�gio financiado pela FAPESP como parte do trabalho de doutorado.
\end{compactitem}


%--------------------------------------------------------------------------
\section*{Participa��o em eventos}
\addcontentsline{toc}{section}{Participa��o em eventos}

\begin{compactitem}
	\item XVII Encontro Nacional de F�sica da Mat�ria Condensada, Po�os de Caldas (2004).
	\item 12th Brazilian Workshop on Semiconductor Physics, S�o Jos� dos Campos (2005).
	\item XVIII Encontro Nacional de F�sica da Mat�ria Condensada, Santos (2005).
\end{compactitem}

%--------------------------------------------------------------------------
\section*{Organiza��o de eventos}
\addcontentsline{toc}{section}{Organiza��o de eventos}

\begin{compactitem}
	\item \LaTeX\ e a Internet, CEPA (2008).
\end{compactitem}

%--------------------------------------------------------------------------
\section*{Projetos de pesquisa}
\addcontentsline{toc}{section}{Projetos de pesquisa}

\begin{compactitem}
	\item Estudo e desenvolvimento dos processos de cria��o e produ��o de Objetos de Aprendizagem para o projeto Univesp, junto a CTI e financiado pela FUSP (projeto Univesp), 2010.
\end{compactitem}

%--------------------------------------------------------------------------
\section*{Cursos e treinamentos produzidos}
\addcontentsline{toc}{section}{Cursos e treinamentos produzidos}

\begin{compactitem}
	\item \textsl{Usando \LaTeX; pensando \TeX}, curso semi-presencial com efoque pr�tico (parte n�o-presencial atrav�s do sistema de gerenciamento de cursos Moodle da CTI, dom�nio \url{moodle.stoa.usp.br}).
	\item \textsl{Aula tem�tica com lousa interativa}, produzida para a abertura oficial do projeto Aulas Interativas, na Secretaria de Estado da Educa��o, em 6 de novembro de 2009.
\end{compactitem}

%--------------------------------------------------------------------------
\section*{Cursos e treinamentos ministrados}
\addcontentsline{toc}{section}{Cursos e treinamentos ministrados}

\begin{compactitem}
	\item \textsl{Usando \LaTeX; pensando \TeX}, CEPA, 15 horas (2008).
	\item \textsl{Usando \LaTeX; pensando \TeX}, CEPA/CTI, 20 horas (2008).
	\item \textsl{Usando \LaTeX; pensando \TeX}, CEPA/CTI, 24 horas (2009).
	\item Treinamento de professores da rede p�blica de Hortol�ndia para a aula tem�tica apresentada na abertura oficial do projeto Aulas Interativas, Diretoria de Ensino de Sumar� (3--5 de novembro de 2009).
\end{compactitem}

%--------------------------------------------------------------------------
\section*{\foreign{Softwares} produzidos}
\addcontentsline{toc}{section}{\foreign{Softwares} produzidos}

\begin{compactitem}
	\item Script LabTalk para a obten��o das mobilidades qu�nticas das sub-bandas de gases bidimensionais de el�trons a partir de oscila��es de magnetoresist�ncia (efeito Shubnikov-de Haas).
\end{compactitem}

%--------------------------------------------------------------------------
\section*{Palestras}
\addcontentsline{toc}{section}{Palestras}

\begin{compactitem}
	\item \textsl{Um modelo de aula interativa com lousa eletr�nica}, CEPA, 19 de maio de 2009 (apresenta��o da proposta do CEPA para o projeto Aulas Interativas, para membros da SEE e da Dell).
	\item \textsl{Um modelo de aula interativa com lousa eletr�nica}, Diretoria de Ensino de Sumar�, julho de 2009 (apresenta��o do projeto Aulas Interativas para os diretores das escolas p�blicas de Hortol�ndia).
	\item \textsl{Tecnologia educacional: atividades interativas para EaD e lousas digitais}, CEPA, 23 de outubro de 2009 [apresenta��o feita para os alunos da disciplina FAP0459 (tecnologia educacional), a convite do Prof. Dr. Ewout ter Haar].
	\item \textsl{Atividades Interativas para EaD e lousas eletr�nicas}, CTI, 13 de julho de 2010 (apresenta��o feita para os coordenadores do projeto Univesp, sobre uma proposta de atividades interativas para o mesmo projeto).
\end{compactitem}

%--------------------------------------------------------------------------
\section*{Orienta��es e supervis�es conclu�das}
\addcontentsline{toc}{section}{Orienta��es e supervis�es conclu�das}

\begin{compactitem}
	\item Coordena��o de uma equipe de 4--6 programadores para a produ��o de \foreign{softwares} educacionais para o projeto Aulas Interativas. De julho de 2009 a dezembro de 2010.
\end{compactitem}

%--------------------------------------------------------------------------
\section*{Orienta��es e supervis�es em andamento}
\addcontentsline{toc}{section}{Orienta��es e supervis�es em andamento}

\begin{compactitem}
	\item Coordena��o de uma equipe de 6 programadores e 1 ilustrador para a produ��o de Objetos de Aprendizagem para os projetos Univesp e Redefor. Em andamento desde agosto de 2010.
\end{compactitem}


	
	\begin{thebibliography}{--}
	\bibitem{curso-LaTeX} Curso \textsl{Usando \LaTeX; pensando \TeX}, de Ivan Ramos Pagnossin, em \url{http://goo.gl/bz6ST} (2011.06.13)
\end{thebibliography}	
	\addcontentsline{toc}{chapter}{Referências}

\end{document}
%-------------------------------------------------------------------------------
